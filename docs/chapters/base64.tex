\chapter{Base64 Encoding and Decoding}

\section{Introduction}
Base64 encoding is a method of converting binary data into a set of 64 printable characters from the ASCII standard. This encoding is commonly used to encode data when there is a requirement to transmit binary data over media that are designed to deal with textual data. The Base64 index table consists of the characters \texttt{A-Z}, \texttt{a-z}, \texttt{0-9}, plus two additional characters, commonly \texttt{+} and \texttt{/}, for a total of 64 characters.

\section{Principles of Base64 Encoding}
Base64 encoding processes data in blocks of 3 bytes (24 bits) at a time. Each block of 24 bits is then divided into four 6-bit groups. Each 6-bit group is used as an index into the Base64 character table, resulting in a 4-character encoding of the original 24-bit block.

Let the input stream be a sequence of bits represented by $\mathbf{b}$. This stream is divided into blocks of 24 bits:
\[\mathbf{b} = b_1 b_2 b_3 \ldots b_{24} b_{25} b_{26} \ldots\]

Each block of 24 bits $(b_1 b_2 \ldots b_{24})$ is then split into four 6-bit groups:
\[\mathbf{g}_1 = b_1 b_2 \ldots b_6, \; \mathbf{g}_2 = b_7 b_8 \ldots b_{12}, \; \mathbf{g}_3 = b_{13} b_{14} \ldots b_{18}, \; \mathbf{g}_4 = b_{19} b_{20} \ldots b_{24}\]

Each group $\mathbf{g}_i$ is then converted into a decimal index $n_i$:
\[n_i = \sum_{k=0}^{5} g_{i,k} \cdot 2^{5-k}\]
where $g_{i,k}$ is the $k$-th bit of group $\mathbf{g}_i$.

Each index $n_i$ is used to select a character from the Base64 table, forming the encoded string.

\section{Principles of Base64 Decoding}
Base64 decoding is the reverse process. Each character in the encoded string is mapped back to its corresponding 6-bit group. These groups are then concatenated to form the original byte sequence.

Given a Base64 encoded string, each character is converted back into a 6-bit group according to the Base64 index table. Let these groups be $\mathbf{d}_1, \mathbf{d}_2, \mathbf{d}_3, \mathbf{d}_4$. The original 24-bit block is reconstructed by concatenating these groups:
\[\mathbf{b}' = \mathbf{d}_1 \mathbf{d}_2 \mathbf{d}_3 \mathbf{d}_4\]

The binary sequence $\mathbf{b}'$ is then divided back into the original bytes to retrieve the original data.

\section{Padding in Base64 Encoding}
In cases where the input byte sequence is not a multiple of 3, padding is used. The input is padded with zeros to form a complete 24-bit block. The corresponding Base64 encoded output is then padded with the '=' character to indicate that padding was used. The number of '=' characters used indicates the number of bytes that were added as padding (one byte of padding results in two '=' characters, and two bytes of padding results in one '=' character).
