\chapter{Block Cipher}
Block ciphers are a fundamental component in cryptographic systems. They transform fixed-size blocks of plaintext into ciphertext using a symmetric key. The transformation is designed to be reversible only with knowledge of the key.

\section{Definition and Structure}

\begin{itemize}
	\item \textbf{Secure Pseudo-Random Permutation (PRP) and Substitution Groups:}
	\begin{itemize}
		\item \textbf{Definition:} A block cipher is considered a secure PRP if it is indistinguishable from a random permutation of the input bits, making it resistant to cryptanalysis.
		\item \textbf{Substitution Groups:} Block ciphers often use substitution-permutation networks (SPNs) that include substitution groups. These groups perform non-linear transformations, crucial for creating cryptographic strength.
	\end{itemize}
	\item \textbf{Confidentiality for Fixed n-bit Data (Blocks):} 
	\begin{itemize}
		\item \textbf{Fixed Block Size:} Block ciphers encrypt and decrypt data in fixed-size blocks (commonly 64 or 128 bits). This fixed size is crucial for the algorithm's structure and security.
		\item \textbf{Padding Schemes:} When the data doesn't fit perfectly into a block, padding schemes are used to fill the remaining space, ensuring consistent block sizes.
	\end{itemize}
	\item \textbf{Block Cipher Operation Modes for Variable-Length Data:} 
	\begin{itemize}
		\item \textbf{Mode of Operation:} To handle variable-length data, block ciphers use different modes of operation like CBC (Cipher Block Chaining), CFB (Cipher Feedback), and GCM (Galois/Counter Mode).
		\item \textbf{Ensuring Security:} Each mode offers distinct features for security and efficiency, often enhancing the cipher's resistance to various attack vectors.
	\end{itemize}
	\item \textbf{Advantages Over Asymmetric Key Cryptography:}
	\begin{itemize}
		\item \textbf{High-Speed Computation:} Block ciphers are generally faster and require less computational power compared to asymmetric key cryptography.
		\item \textbf{Suitability:} This makes them suitable for encrypting large volumes of data and in environments with limited resources.
	\end{itemize}
	\item \textbf{Deriving Other Cryptographic Functions:}
	\begin{itemize}
		\item \textbf{Versatility:} Block ciphers can be used to design other cryptographic functions like hash functions, message authentication codes (MACs), and random number generators.
		\item \textbf{Construction Techniques:} Techniques like Cipher Block Chaining-MAC (CBC-MAC) and Counter mode (CTR) are examples of how block ciphers can be adapted for these purposes.
	\end{itemize}
\end{itemize}

Block ciphers are a critical element in the cryptographic landscape, providing a versatile and efficient means for securing digital data. Their adaptability and robustness make them an indispensable tool in the design of secure communication protocols and cryptographic systems.


\section{Modes of Operations}

\begin{table}[h!]\centering\renewcommand{\arraystretch}{1.05} % Increase row height by 1.5 times
	\caption{Comparison of Modes}
	\begin{tabular*}{\textwidth}{@{\extracolsep{\fill}}cccccccc}
		\toprule[1.2pt]
		Mode & Integrity & Authentication & $\EncryptBlk$ & $\DecryptBlk$ & Padding & IV & $\abs{P}\overset{?}{=}\abs{C}$ \\
		\midrule
		$\ECB$ & \yes & \no & \yes & \yes & \yes & \no & $\abs{P}<\abs{C}$ \\
		$\CBC$ & \yes & \no & \yes & \yes & \yes & \yes & $\abs{P}<\abs{C}$ \\
		$\OFB$ & \yes & \no & \yes & \no & \no & \yes & $\abs{P}=\abs{C}$ \\
		$\CFB$ & \yes & \no & \yes & \no & \no & \yes & $\abs{P}=\abs{C}$ \\
		$\CTR$ & \yes & \no & \yes & \no & \no & \yes & $\abs{P}=\abs{C}$ \\
		$\CBCCS$ & \yes & \no & \yes & \yes & \no & \yes & $\abs{P}=\abs{C}$ \\
		\bottomrule[1.2pt]
	\end{tabular*}
\end{table}

\subsection{Padding}
Block ciphers require input lengths to be a multiple of the block size. Padding is used to extend the last block of plaintext to the required length. Without proper padding, the encryption process may be insecure or infeasible.\\
\\
There are several padding schemes used in practice, such as:
\begin{table}[ht]
	\centering\renewcommand{\arraystretch}{1.05}
	\caption{Padding Standards in Block Ciphers}
	\begin{tabular*}{\textwidth}{>{\bfseries}l l}
		\toprule
		\textbf{Standard Name} & \textbf{Padding Method} \\
		\midrule
		\multirow{2}{*}{PKCS\#7} & Pad with bytes all the same value as the number of padding bytes \\
		& \texttt{\dots dd | dd dd dd dd dd dd dd dd dd dd dd dd \textcolor{red}{04} \textcolor{red}{04} \textcolor{red}{04} \textcolor{red}{04}} | \\
		\hline
		\multirow{2}{*}{ANSI X9.23} & Pad with zeros, last byte is the number of padding bytes \\
		& \texttt{\dots dd | dd dd dd dd dd dd dd dd dd dd dd \textcolor{red}{00} \textcolor{red}{00} \textcolor{red}{00} \textcolor{red}{00} \textcolor{red}{05}} | \\
		\hline
		\multirow{2}{*}{ISO/IEC 7816-4} & First byte is '80' (hex), followed by zeros \\
		& \texttt{\dots dd | dd dd dd dd dd dd dd dd dd dd \textcolor{red}{80} \textcolor{red}{00} \textcolor{red}{00} \textcolor{red}{00} \textcolor{red}{00} \textcolor{red}{00}} | \\
		\hline
		\multirow{2}{*}{ISO 10126} & Pad with random bytes, last byte is the number of padding bytes \\
		& \texttt{\dots dd | dd dd dd dd dd dd dd dd dd dd \textcolor{red}{2e} \textcolor{red}{49} \textcolor{red}{1b} \textcolor{red}{c1} \textcolor{red}{aa} \textcolor{red}{06}} | \\
		\bottomrule
	\end{tabular*}
	\label{tab:padding_standards}
\end{table}

\newpage

\subsection{$\ECB$ (Electronic CodeBook)}
\begin{algorithm}[H]
	\caption{Electronic CodeBook}
	\begin{multicols}{2}
		\KwIn{$K$ and $P=P_1\parallel\cdots\parallel P_N$ ($P_i\in\binaryfield^{n}$)}
		\KwOut{$C=C_1\parallel\cdots\parallel C_N$ ($C_i\in\binaryfield^n$)}
		\BlankLine
		\For{$i\gets 1$ \KwTo $N$}{
			$C_i\gets\EncryptBlk(K, P_i)$\;
		}
		\Return $C=C_1\parallel\cdots\parallel C_N$\;
		\columnbreak % Move to the next column
		\setcounter{AlgoLine}{0}  % Reset line numbering
		\KwIn{$K$ and $C=C_1\parallel\cdots\parallel C_N$ ($C_i\in\binaryfield^n$)}
		\KwOut{$P=P_1\parallel\cdots\parallel P_N$ ($P_i\in\binaryfield^{n}$)}
		\BlankLine
		\For{$i\gets 1$ \KwTo $N$}{
			$P_i\gets\DecryptBlk(K, C_i)$\;
		}
		\Return $C=C_1\parallel\cdots\parallel C_N$\;
	\end{multicols}
	\BlankLine
\end{algorithm}
\vspace{12pt}
\begin{center}
\begin{minipage}{.49\textwidth}\centering
\begin{tikzpicture}
	% Define styles
	\tikzset{
		block/.style={rectangle, draw, fill=blue!20, text width=2em, text centered, minimum height=2.5em},
		line/.style={draw, -Latex}
	}
	
	% Nodes
	\node[block] (p1) {$P_1$};
	\node[block, right=of p1] (p2) {$P_2$};
	\node[block, right=of p2] (p3) {...};
	\node[block, right=of p3] (p4) {$P_N$};
	
	\node[block, below=of p1] (e1) {$E_K$};
	\node[block, below=of p2] (e2) {$E_K$};
	\node[block, below=of p3] (e3) {...};
	\node[block, below=of p4] (e4) {$E_K$};
	
	\node[block, below=of e1] (c1) {$C_1$};
	\node[block, below=of e2] (c2) {$C_2$};
	\node[block, below=of e3] (c3) {...};
	\node[block, below=of e4] (c4) {$C_N$};
	
	% Arrows
	\path [line] (p1) -- (e1);
	\path [line] (p2) -- (e2);
	\path [line] (p3) -- (e3);
	\path [line] (p4) -- (e4);
	
	\path [line] (e1) -- (c1);
	\path [line] (e2) -- (c2);
	\path [line] (e3) -- (c3);
	\path [line] (e4) -- (c4);
\end{tikzpicture}
\end{minipage}
\begin{minipage}{.49\textwidth}\centering
\begin{tikzpicture}
	% Define styles
	\tikzset{
		block/.style={rectangle, draw, fill=red!20, text width=2em, text centered, minimum height=2.5em},
		line/.style={draw, -Latex}
	}
	
	% Nodes
	\node[block] (p1) {$P_1$};
	\node[block, right=of p1] (p2) {$P_2$};
	\node[block, right=of p2] (p3) {...};
	\node[block, right=of p3] (p4) {$P_N$};
	
	\node[block, below=of p1] (e1) {$E_K$};
	\node[block, below=of p2] (e2) {$E_K$};
	\node[block, below=of p3] (e3) {...};
	\node[block, below=of p4] (e4) {$E_K$};
	
	\node[block, below=of e1] (c1) {$C_1$};
	\node[block, below=of e2] (c2) {$C_2$};
	\node[block, below=of e3] (c3) {...};
	\node[block, below=of e4] (c4) {$C_N$};
	
	% Arrows
	\path [line] (e1) -- (p1);
	\path [line] (e2) -- (p2);
	\path [line] (e3) -- (p3);
	\path [line] (e4) -- (p4);
	
	\path [line] (c1) -- (e1);
	\path [line] (c2) -- (e2);
	\path [line] (c3) -- (e3);
	\path [line] (c4) -- (e4);
\end{tikzpicture}
\end{minipage}

\end{center}

\vspace{24pt}
\begin{remark}
\ \begin{enumerate}[(1)]
	\item For a pair of plaintext/ciphertext $(P, C)$ and $(P', C')$, \[
	P=P'\implies C=C'.
	\]
	\item A single-bit error in the ciphertext affects only the corresponding block in the decryption.
	\item Block order changes, as well as additions/deletions, are possible. Therefore, to ensure the integrity of the ciphertext (i.e., detection or prevention of tampering), it should be used in conjunction with a checksum or a Message Authentication Code (MAC).
\end{enumerate}
\end{remark}

\newpage
\subsection{$\CBC$ (Cipher Block Chaining)}
\begin{algorithm}[H]
	\caption{Cipher Block Chaining}
	\begin{multicols}{2}
		\KwIn{$K$, $IV$ and $P=P_1\parallel\cdots\parallel P_N$}
		\KwOut{$C=C_1\parallel\cdots\parallel C_N$}
		\BlankLine
		$C_0\gets IV$\;
		\For{$i\gets 1$ \KwTo $N$}{
			$C_i\gets\EncryptBlk(K, P_i\oplus C_{i-1})$\;
		}
		\Return $C=C_1\parallel\cdots\parallel C_N$\;
		\columnbreak % Move to the next column
		\setcounter{AlgoLine}{0}  % Reset line numbering
		\KwIn{$K$, $IV$ and $C=C_1\parallel\cdots\parallel C_N$}
		\KwOut{$P=P_1\parallel\cdots\parallel P_N$}
		\BlankLine
		$C_0\gets IV$\;
		\For{$i\gets 1$ \KwTo $N$}{
			$P_i\gets C_{i-1}\oplus\DecryptBlk(K, C_{i})$\;
		}
		\Return $P=P_1\parallel\cdots\parallel P_N$\;
	\end{multicols}
	\BlankLine
\end{algorithm}

\begin{center}
\begin{minipage}{.49\textwidth}\centering
\begin{tikzpicture}
	% Define styles
	\tikzset{
		block/.style={rectangle, draw, fill=blue!20, text width=1em, text centered, minimum height=1em},
		line/.style={draw, -Latex},
		xor/.style={circle, draw, minimum size=0cm, inner sep=0pt}
	}
	% Nodes
	\node[block] (iv) {IV};
	\node[block, right=1cm of iv] (p1) {$P_1$};
	\node[block, right=1cm of p1] (p2) {$P_2$};
	\node[block, right=1cm of p2] (p3) {...};
	\node[block, right=1cm of p3] (p4) {$P_n$};
	
	\node[below=1cm of p1] (xor1) {\Huge $\oplus$};
	\node[below=1cm of p2] (xor2) {\Huge $\oplus$};
	\node[below=1cm of p3] (xor3) {\Huge $\oplus$};
	\node[below=1cm of p4] (xor4) {\Huge $\oplus$};
	
	\node[block, below=1cm of xor1] (e1) {$E_K$};
	\node[block, below=1cm of xor2] (e2) {$E_K$};
	\node[block, below=1cm of xor3] (e3) {...};
	\node[block, below=1cm of xor4] (e4) {$E_K$};
	
	\node[block, below=1cm of e1] (c1) {$C_1$};
	\node[block, below=1cm of e2] (c2) {$C_2$};
	\node[block, below=1cm of e3] (c3) {...};
	\node[block, below=1cm of e4] (c4) {$C_n$};
	
	% Arrows
	\draw[line] (iv.south) to[bend right=45] (xor1.west);
	\draw[line] (p1) -- (xor1);
	\draw[line] (xor1) -- (e1);
	\draw[line] (e1) -- (c1);
	
	\draw[line] (e1.south) to[bend right=90] ($(c1)!.6!(e2)$) to[bend left=60] (xor2.west);
	\draw[line] (p2) -- (xor2);
	\draw[line] (xor2) -- (e2);
	\draw[line] (e2) -- (c2);
	
	\draw[line] (c2.east) to[bend right=45] ($(c2)!.5!(xor3)$) to[bend left=45] (xor3.west);
	\draw[line] (p3) -- (xor3);
	\draw[line] (xor3) -- (e3);
	\draw[line] (e3) -- (c3);
	
	\draw[line] (c3.east) to[bend right=45] ($(c3)!.5!(xor4)$) to[bend left=45] (xor4.west);
	\draw[line] (p4) -- (xor4);
	\draw[line] (xor4) -- (e4);
	\draw[line] (e4) -- (c4);
\end{tikzpicture}
\end{minipage}
\begin{minipage}{.49\textwidth}\centering
	\begin{tikzpicture}
		% Define styles
		\tikzset{
			block/.style={rectangle, draw, fill=blue!20, text width=1em, text centered, minimum height=1em},
			line/.style={draw, -Latex},
			xor/.style={circle, draw, minimum size=0cm, inner sep=0pt}
		}
		
		% Nodes
		\node[block] (iv) {IV};
		\node[block, right=1cm of iv] (p1) {$P_1$};
		\node[block, right=1cm of p1] (p2) {$P_2$};
		\node[block, right=1cm of p2] (p3) {...};
		\node[block, right=1cm of p3] (p4) {$P_n$};
		
		\node[below=1cm of p1] (xor1) {\Huge $\oplus$};
		\node[below=1cm of p2] (xor2) {\Huge $\oplus$};
		\node[below=1cm of p3] (xor3) {\Huge $\oplus$};
		\node[below=1cm of p4] (xor4) {\Huge $\oplus$};
		
		\node[block, below=1cm of xor1] (e1) {$E_K$};
		\node[block, below=1cm of xor2] (e2) {$E_K$};
		\node[block, below=1cm of xor3] (e3) {...};
		\node[block, below=1cm of xor4] (e4) {$E_K$};
		
		\node[block, below=1cm of e1] (c1) {$C_1$};
		\node[block, below=1cm of e2] (c2) {$C_2$};
		\node[block, below=1cm of e3] (c3) {...};
		\node[block, below=1cm of e4] (c4) {$C_n$};
		
		% Arrows
		\draw[line] (iv.south) to[bend right=45] (xor1.west);
		\draw[line] (xor1) -- (p1);
		\draw[line] (e1) -- (xor1);
		\draw[line] (c1) -- (e1);
		
		\draw[line] (c1.east) to[bend right=45] ($(c1)!.5!(xor2)$) to[bend left=45] (xor2.west);
		\draw[line] (xor2) -- (p2);
		\draw[line] (e2) -- (xor2);
		\draw[line] (c2) -- (e2);
		
		\draw[line] (c2.east) to[bend right=45] ($(c2)!.5!(xor3)$) to[bend left=45] (xor3.west);
		\draw[line] (xor3) -- (p3);
		\draw[line] (e3) -- (xor3);
		\draw[line] (c3) -- (e3);
		
		\draw[line] (c3.east) to[bend right=45] ($(c3)!.5!(xor4)$) to[bend left=45] (xor4.west);
		\draw[line] (xor4) -- (p4);
		\draw[line] (e4) -- (xor4);
		\draw[line] (c4) -- (e4);
	\end{tikzpicture}
\end{minipage}
\end{center}

\vspace{24pt}
\begin{remark}
\ \begin{enumerate}[(1)]
	\item For a set of plaintext/ciphertext pairs $(IV, P, C)$ and $(IV', P', C')$, \[
	IV=IV'\land P=P'\implies C=C'.
	\]
	\item  A one-bit error in the ciphertext will affect the corresponding block and the subsequent block in the decryption process, thereby enabling the detection of such errors.
	\item Altering the order of blocks, as well as adding or deleting them, is not possible.
\end{enumerate}
\end{remark}
\newpage
\subsection{$\OFB$ (Output FeedBack)}
\begin{center}
\begin{tikzpicture}
	% Define styles
	\tikzset{
		block/.style={rectangle, draw, fill=blue!20, text width=2.5em, text centered, minimum height=2.5em},
		line/.style={draw, -Latex},
		xor/.style={circle, draw, minimum size=0.75cm, inner sep=0pt}
	}
	
	% Nodes
	\node[block] (iv) {IV};
	\node[block, right=1cm of iv] (e1) {$E$};
	\node[block, right=1cm of e1] (e2) {$E$};
	\node[block, right=1cm of e2] (e3) {...};
	\node[block, right=1cm of e3] (e4) {$E$};
	
	\node[block, above=1cm of e1] (p1) {P1};
	\node[block, above=1cm of e2] (p2) {P2};
	\node[block, above=1cm of e3] (p3) {...};
	\node[block, above=1cm of e4] (p4) {Pn};
	
	\node[block, below=1cm of e1] (c1) {C1};
	\node[block, below=1cm of e2] (c2) {C2};
	\node[block, below=1cm of e3] (c3) {...};
	\node[block, below=1cm of e4] (c4) {Cn};
	
	\node[xor, right=0.5cm of p1] (xor1) {\large $\oplus$};
	\node[xor, right=0.5cm of p2] (xor2) {\large $\oplus$};
	\node[xor, right=0.5cm of p3] (xor3) {\large $\oplus$};
	\node[xor, right=0.5cm of p4] (xor4) {\large $\oplus$};
	
	% Arrows
	\draw[line] (iv) -- (e1);
	\draw[line] (e1) -- (e2);
	\draw[line] (e2) -- (e3);
	\draw[line] (e3) -- (e4);
	
	\draw[line] (e1) -- (xor1);
	\draw[line] (e2) -- (xor2);
	\draw[line] (e3) -- (xor3);
	\draw[line] (e4) -- (xor4);
	
	\draw[line] (p1) -- (xor1);
	\draw[line] (p2) -- (xor2);
	\draw[line] (p3) -- (xor3);
	\draw[line] (p4) -- (xor4);
	
	\draw[line] (xor1) -- (c1);
	\draw[line] (xor2) -- (c2);
	\draw[line] (xor3) -- (c3);
	\draw[line] (xor4) -- (c4);
	
\end{tikzpicture}
\end{center}

\subsection{$\CFB$ (Ciphertext FeedBack)}
\begin{center}
\begin{tikzpicture}
	% Define styles
	\tikzset{
		block/.style={rectangle, draw, fill=blue!20, text width=2.5em, text centered, minimum height=2.5em},
		line/.style={draw, -Latex},
		xor/.style={circle, draw, minimum size=0.75cm, inner sep=0pt}
	}
	
	% Nodes
	\node[block] (iv) {IV};
	\node[block, right=1cm of iv] (e1) {$E$};
	\node[block, right=1cm of e1] (e2) {$E$};
	\node[block, right=1cm of e2] (e3) {...};
	\node[block, right=1cm of e3] (e4) {$E$};
	
	\node[block, above=1cm of e1] (p1) {P1};
	\node[block, above=1cm of e2] (p2) {P2};
	\node[block, above=1cm of e3] (p3) {...};
	\node[block, above=1cm of e4] (p4) {Pn};
	
	\node[block, below=1cm of e1] (c1) {C1};
	\node[block, below=1cm of e2] (c2) {C2};
	\node[block, below=1cm of e3] (c3) {...};
	\node[block, below=1cm of e4] (c4) {Cn};
	
	\node[xor, right=0.5cm of iv] (xor1) {\large $\oplus$};
	\node[xor, right=0.5cm of e1] (xor2) {\large $\oplus$};
	\node[xor, right=0.5cm of e2] (xor3) {\large $\oplus$};
	\node[xor, right=0.5cm of e3] (xor4) {\large $\oplus$};
	
	% Arrows
	\draw[line] (iv) -- (xor1);
	\draw[line] (xor1) -- (e1);
	\draw[line] (e1) -- (xor2);
	\draw[line] (xor2) -- (e2);
	\draw[line] (e2) -- (xor3);
	\draw[line] (xor3) -- (e3);
	\draw[line] (e3) -- (xor4);
	\draw[line] (xor4) -- (e4);
	
	\draw[line] (p1) -- (xor1);
	\draw[line] (p2) -- (xor2);
	\draw[line] (p3) -- (xor3);
	\draw[line] (p4) -- (xor4);
	
	\draw[line] (e1) -- (c1);
	\draw[line] (e2) -- (c2);
	\draw[line] (e3) -- (c3);
	\draw[line] (e4) -- (c4);
	
\end{tikzpicture}
\end{center}

\subsection{$\CTR$ (CounTeR)}
\begin{center}
\begin{tikzpicture}
	% Define styles
	\tikzset{
		block/.style={rectangle, draw, fill=blue!20, text width=2.5em, text centered, minimum height=2.5em},
		line/.style={draw, -Latex},
		xor/.style={circle, draw, minimum size=0.75cm, inner sep=0pt},
		counter/.style={rectangle, draw, fill=red!20, text width=2.5em, text centered, minimum height=2.5em}
	}
	
	% Nodes
	\node[counter] (iv) {IV};
	\node[block, right=1cm of iv] (e1) {$E$};
	\node[counter, right=1cm of e1] (cnt1) {Cnt+1};
	\node[block, right=1cm of cnt1] (e2) {$E$};
	\node[counter, right=1cm of e2] (cnt2) {Cnt+2};
	\node[block, right=1cm of cnt2] (e3) {...};
	
	\node[block, above=1cm of e1] (p1) {P1};
	\node[block, above=1cm of e2] (p2) {P2};
	\node[block, above=1cm of e3] (p3) {...};
	
	\node[block, below=1cm of e1] (c1) {C1};
	\node[block, below=1cm of e2] (c2) {C2};
	\node[block, below=1cm of e3] (c3) {...};
	
	\node[xor, right=0.5cm of p1] (xor1) {\large $\oplus$};
	\node[xor, right=0.5cm of p2] (xor2) {\large $\oplus$};
	\node[xor, right=0.5cm of p3] (xor3) {\large $\oplus$};
	
	% Arrows
	\draw[line] (iv) -- (e1);
	\draw[line] (cnt1) -- (e2);
	\draw[line] (cnt2) -- (e3);
	
	\draw[line] (e1) -- (xor1);
	\draw[line] (e2) -- (xor2);
	\draw[line] (e3) -- (xor3);
	
	\draw[line] (p1) -- (xor1);
	\draw[line] (p2) -- (xor2);
	\draw[line] (p3) -- (xor3);
	
	\draw[line] (xor1) -- (c1);
	\draw[line] (xor2) -- (c2);
	\draw[line] (xor3) -- (c3);
	
\end{tikzpicture}
\end{center}

\begin{center}



\tikzset{every picture/.style={line width=0.75pt}} %set default line width to 0.75pt        

\begin{tikzpicture}[x=0.75pt,y=0.75pt,yscale=-1,xscale=1]
	%uncomment if require: \path (0,453); %set diagram left start at 0, and has height of 453
	
	\draw   (136,319.33) .. controls (136,311.05) and (142.72,304.33) .. (151,304.33) .. controls (159.28,304.33) and (166,311.05) .. (166,319.33) .. controls (166,327.62) and (159.28,334.33) .. (151,334.33) .. controls (142.72,334.33) and (136,327.62) .. (136,319.33) -- cycle ; \draw   (136,319.33) -- (166,319.33) ; \draw   (151,304.33) -- (151,334.33) ;
	\draw   (336,316) .. controls (336,307.72) and (342.72,301) .. (351,301) .. controls (359.28,301) and (366,307.72) .. (366,316) .. controls (366,324.28) and (359.28,331) .. (351,331) .. controls (342.72,331) and (336,324.28) .. (336,316) -- cycle ; \draw   (336,316) -- (366,316) ; \draw   (351,301) -- (351,331) ;
	\draw   (536.67,320) .. controls (536.67,311.72) and (543.38,305) .. (551.67,305) .. controls (559.95,305) and (566.67,311.72) .. (566.67,320) .. controls (566.67,328.28) and (559.95,335) .. (551.67,335) .. controls (543.38,335) and (536.67,328.28) .. (536.67,320) -- cycle ; \draw   (536.67,320) -- (566.67,320) ; \draw   (551.67,305) -- (551.67,335) ;
	%Shape: Rectangle [id:dp23698905140771065] 
	\draw   (92.5,164.67) -- (209.5,164.67) -- (209.5,236.67) -- (92.5,236.67) -- cycle ;
	%Shape: Rectangle [id:dp060189272097281865] 
	\draw   (292.5,164.67) -- (409.5,164.67) -- (409.5,236.67) -- (292.5,236.67) -- cycle ;
	%Shape: Rectangle [id:dp7358028365384404] 
	\draw   (492.17,164.67) -- (609.17,164.67) -- (609.17,236.67) -- (492.17,236.67) -- cycle ;
	%Straight Lines [id:da6855921879025784] 
	\draw    (150.33,237.33) -- (150.97,300.4) ;
	\draw [shift={(151,303.4)}, rotate = 269.42] [fill={rgb, 255:red, 0; green, 0; blue, 0 }  ][line width=0.08]  [draw opacity=0] (8.93,-4.29) -- (0,0) -- (8.93,4.29) -- cycle    ;
	%Straight Lines [id:da8393200361220992] 
	\draw    (351,237.33) -- (351.64,300.4) ;
	\draw [shift={(351.67,303.4)}, rotate = 269.42] [fill={rgb, 255:red, 0; green, 0; blue, 0 }  ][line width=0.08]  [draw opacity=0] (8.93,-4.29) -- (0,0) -- (8.93,4.29) -- cycle    ;
	%Straight Lines [id:da812579829875762] 
	\draw    (550.67,237.33) -- (551.3,300.4) ;
	\draw [shift={(551.33,303.4)}, rotate = 269.42] [fill={rgb, 255:red, 0; green, 0; blue, 0 }  ][line width=0.08]  [draw opacity=0] (8.93,-4.29) -- (0,0) -- (8.93,4.29) -- cycle    ;
	%Straight Lines [id:da8953409064226481] 
	\draw    (65,320.33) -- (134,320.33) ;
	\draw [shift={(137,320.33)}, rotate = 180] [fill={rgb, 255:red, 0; green, 0; blue, 0 }  ][line width=0.08]  [draw opacity=0] (8.93,-4.29) -- (0,0) -- (8.93,4.29) -- cycle    ;
	%Straight Lines [id:da29517048533379] 
	\draw    (150.33,79.8) -- (150.98,161.67) ;
	\draw [shift={(151,164.67)}, rotate = 269.55] [fill={rgb, 255:red, 0; green, 0; blue, 0 }  ][line width=0.08]  [draw opacity=0] (8.93,-4.29) -- (0,0) -- (8.93,4.29) -- cycle    ;
	%Straight Lines [id:da14835328393275948] 
	\draw    (351,79.8) -- (351.64,161.67) ;
	\draw [shift={(351.67,164.67)}, rotate = 269.55] [fill={rgb, 255:red, 0; green, 0; blue, 0 }  ][line width=0.08]  [draw opacity=0] (8.93,-4.29) -- (0,0) -- (8.93,4.29) -- cycle    ;
	%Straight Lines [id:da5864783489550343] 
	\draw    (550,79.8) -- (550.64,161.67) ;
	\draw [shift={(550.67,164.67)}, rotate = 269.55] [fill={rgb, 255:red, 0; green, 0; blue, 0 }  ][line width=0.08]  [draw opacity=0] (8.93,-4.29) -- (0,0) -- (8.93,4.29) -- cycle    ;
	%Straight Lines [id:da997385537510096] 
	\draw    (150.33,332.67) -- (150.97,395.67) ;
	\draw [shift={(151,398.67)}, rotate = 269.42] [fill={rgb, 255:red, 0; green, 0; blue, 0 }  ][line width=0.08]  [draw opacity=0] (8.93,-4.29) -- (0,0) -- (8.93,4.29) -- cycle    ;
	%Straight Lines [id:da9215134990511367] 
	\draw    (351,332.67) -- (351.64,395.67) ;
	\draw [shift={(351.67,398.67)}, rotate = 269.42] [fill={rgb, 255:red, 0; green, 0; blue, 0 }  ][line width=0.08]  [draw opacity=0] (8.93,-4.29) -- (0,0) -- (8.93,4.29) -- cycle    ;
	%Straight Lines [id:da6364900722654636] 
	\draw    (550.67,332.67) -- (551.3,395.67) ;
	\draw [shift={(551.33,398.67)}, rotate = 269.42] [fill={rgb, 255:red, 0; green, 0; blue, 0 }  ][line width=0.08]  [draw opacity=0] (8.93,-4.29) -- (0,0) -- (8.93,4.29) -- cycle    ;
	%Straight Lines [id:da46397299472270315] 
	\draw    (265,315.33) -- (334,315.33) ;
	\draw [shift={(337,315.33)}, rotate = 180] [fill={rgb, 255:red, 0; green, 0; blue, 0 }  ][line width=0.08]  [draw opacity=0] (8.93,-4.29) -- (0,0) -- (8.93,4.29) -- cycle    ;
	%Straight Lines [id:da48196890203542964] 
	\draw    (465,319.33) -- (534,319.33) ;
	\draw [shift={(537,319.33)}, rotate = 180] [fill={rgb, 255:red, 0; green, 0; blue, 0 }  ][line width=0.08]  [draw opacity=0] (8.93,-4.29) -- (0,0) -- (8.93,4.29) -- cycle    ;
	
	% Text Node
	\draw (133,53.07) node [anchor=north west][inner sep=0.75pt]    {$CTR$};
	% Text Node
	\draw (319.5,53.07) node [anchor=north west][inner sep=0.75pt]    {$CTR+1$};
	% Text Node
	\draw (519.17,53.07) node [anchor=north west][inner sep=0.75pt]    {$CTR+2$};
	% Text Node
	\draw (40.67,312.73) node [anchor=north west][inner sep=0.75pt]    {$P_{1}$};
	% Text Node
	\draw (140,192.07) node [anchor=north west][inner sep=0.75pt]    {$E_{K}$};
	% Text Node
	\draw (340,195.07) node [anchor=north west][inner sep=0.75pt]    {$E_{K}$};
	% Text Node
	\draw (539.67,192.07) node [anchor=north west][inner sep=0.75pt]    {$E_{K}$};
	% Text Node
	\draw (140.5,403.07) node [anchor=north west][inner sep=0.75pt]    {$C_{1}$};
	% Text Node
	\draw (340.5,403.07) node [anchor=north west][inner sep=0.75pt]    {$C_{2}$};
	% Text Node
	\draw (540.17,403.07) node [anchor=north west][inner sep=0.75pt]    {$C_{3}$};
	% Text Node
	\draw (240.67,307.73) node [anchor=north west][inner sep=0.75pt]    {$P_{2}$};
	% Text Node
	\draw (440.67,311.73) node [anchor=north west][inner sep=0.75pt]    {$P_{3}$};
\end{tikzpicture}
\end{center}

\subsection{$\CBCCS$ (Ciphertext Stealing)}
\begin{center}
\begin{tikzpicture}
	% Define styles
	\tikzset{
		block/.style={rectangle, draw, fill=blue!20, text width=2.5em, text centered, minimum height=2.5em},
		line/.style={draw, -Latex},
		xor/.style={circle, draw, minimum size=0.75cm, inner sep=0pt}
	}
	
	% Nodes
	\node[block] (iv) {IV};
	\node[block, right=of iv] (p1) {P1};
	\node[block, right=of p1] (p2) {P2};
	\node[block, right=of p2] (p3) {...};
	\node[block, right=of p3] (pn1) {Pn-1};
	\node[block, right=of pn1, fill=red!20] (pn) {Pn};
	
	\node[block, below=of p1] (e1) {E};
	\node[block, below=of p2] (e2) {E};
	\node[block, below=of p3] (e3) {...};
	\node[block, below=of pn1, fill=red!20] (en1) {E};
	\node[block, below=of pn, fill=red!20] (en) {E};
	
	\node[block, below=of e1] (c1) {C1};
	\node[block, below=of e2] (c2) {C2};
	\node[block, below=of e3] (c3) {...};
	\node[block, below=of en1, fill=red!20] (cn1) {Cn-1};
	\node[block, below=of en, fill=red!20] (cn) {Cn};
	
	% Arrows
	\draw[line] (p1) -- (e1);
	\draw[line] (p2) -- (e2);
	\draw[line] (p3) -- (e3);
	\draw[line] (pn1) -- (en1);
	\draw[line] (pn) -- (en);
	
	\draw[line] (iv) |- ($(e1)!0.5!(c1)$) -| (e1);
	\draw[line] (c1) |- ($(e2)!0.5!(c2)$) -| (e2);
	\draw[line] (c2) |- ($(e3)!0.5!(c3)$) -| (e3);
	\draw[line] (c3) |- ($(en1)!0.5!(cn1)$) -| (en1);
	\draw[line] (cn1) |- ($(en)!0.5!(cn)$) -| (en);
	
	\draw[line] (e1) -- (c1);
	\draw[line] (e2) -- (c2);
	\draw[line] (e3) -- (c3);
	\draw[line] (en1) -- (cn1);
	\draw[line] (en) -- (cn);
	
	% Cipher stealing notation
	\draw[line, dashed, red] (cn1.north) |- ($(en)!0.5!(cn)$) -| (en.north);
	
\end{tikzpicture}
\end{center}
