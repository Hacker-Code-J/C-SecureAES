\chapter{Block Cipher AES-128}

\section{Overview of AES-128}

\begin{itemize}
	\item $\text{KeyExpansion}:\binaryfield^{128}\to\binaryfield^{1408=4\cdot(10+1)\cdot32}$.
	\item $\text{AddRoundKey}:\binaryfield^{128}\times\binaryfield^{128}\to\binaryfield^{128}$.
	\item $\text{SubBytes/ShiftRows/MixColumns}:\binaryfield^{128}\to\binaryfield^{128}$.
\end{itemize}

\begin{algorithm}[H]
	\caption{Encryption of AES-128}
	
	\KwIn{block $\mathsf{src} \in \binaryfield^{128}$, round-keys $\{rk_i\}_{i=0}^{11}$ ($rk_i \in \binaryfield^{128}$)}
	\KwOut{block $\mathsf{dst} \in \binaryfield^{128}$}
	\BlankLine
	$t \leftarrow \text{AddRoundKey}(\mathsf{src}, rk_0)$\;
	\For{$i \leftarrow 1$ \KwTo $9$}{
		$t \leftarrow (\text{MixColumns}\circ\text{ShiftRows}\circ\text{SubBytes})(t)$\;
		$t \leftarrow \text{AddRoundKey}(t, rk_i)$\;
	}
	$t \leftarrow (\text{ShiftRows}\circ\text{SubBytes})(t)$\;
	$t \leftarrow \text{AddRoundKey}(t, rk_{10})$\;
	$\mathsf{dst} \leftarrow t$\;
	\Return{$\mathsf{dst}$}\;
\end{algorithm}

\begin{algorithm}[H]
	\caption{Decryption of AES-128}
	
	\KwIn{block $\mathsf{src} \in \binaryfield^{128}$, round-keys $\{rk_i\}_{i=0}^{11}$ ($rk_i \in \binaryfield^{128}$)}
	\KwOut{block $\mathsf{dst} \in \binaryfield^{128}$}
	\BlankLine
	$t \leftarrow \text{AddRoundKey}(\mathsf{src}, rk_{10})$\;
	\For{$i \leftarrow 9$ \KwTo $1$}{
		$t \leftarrow (\text{InvSubBytes}\circ\text{InvShiftRows})(t)$\;
		$t \leftarrow \text{AddRoundKey}(t, rk_i)$\;
		$t \leftarrow \text{InvMixColumns}(t)$\;
	}
	$t \leftarrow (\text{InvShiftRows}\circ\text{InvSubBytes})(t)$\;
	$t \leftarrow \text{AddRoundKey}(t, rk_{0})$\;
	$\mathsf{dst} \leftarrow t$\;
	\Return{$\mathsf{dst}$}\;
\end{algorithm}

\newpage
\section{Functions and Constants used in AES}
\subsection{Key Expansion}
\begin{itemize}
	\item \hl{$\text{RotWord}:\binaryfield^{32}\to\binaryfield^{32}$} is defined by \[
	\text{RotWord}\left(X_0\parallel X_1\parallel X_2\parallel X_3\right):=X_1\parallel X_2\parallel X_3\parallel X_0\quad\text{for}\quad X_i\in\binaryfield^8.
	\]
	\begin{lstlisting}[style=C, caption={RotWord rotates the input word left by one byte},captionpos=t]
u32 RotWord(u32 word) {
	return (word << 0x08) | (word >> 0x18);
}
	\end{lstlisting}
	\item \hl{$\text{SubWord}:\binaryfield^{32}\to\binaryfield^{32}$} is defined by \[
	\text{SubWord}(X_0\parallel X_1\parallel X_2\parallel X_3):=s(X_0)\parallel s(X_1)\parallel s(X_2)\parallel s(X_3)\quad\text{for}\quad X_i\in\binaryfield^8.
	\] Here, $s:{\binaryfield^8}\to{\binaryfield^8}$ is the \hyperlink{s-box}{S-box}.
	\begin{lstlisting}[style=C, caption={SubWord applies the S-box to each byte of the input word},captionpos=t]
u32 SubWord(u32 word) {
	return (u32)s_box[word >> 0x18] << 0x18 | 
		(u32)s_box[(word >> 0x10) & 0xFF] << 0x10 | 
		(u32)s_box[(word >> 0x08) & 0xFF] << 0x08 | 
		(u32)s_box[word & 0xFF];
}
	\end{lstlisting}
	\item \hl{Round Constant $\texttt{rCon}$}:
	
	The constant $\texttt{rCon}_i\in\F_{2^8}$ used in generating the $i$-th round key corresponds to the value of $x^{i-1}$ in the binary finite field $\F_{2^8}$ and is as follows:
	\begin{table}[h!]\centering\renewcommand{\arraystretch}{1.5} % Increase row height by 1.5 times
		\begin{tabular*}{\textwidth}{@{\extracolsep{\fill}}c||cccccccccc}
			$i$ & 1 & 2 & 3 & 4 & 5 & 6 & 7 & 8 & 9 & 10\\
			\hline
			$\texttt{Rcon}_i$ & $\texttt{0x01}$ & $\texttt{0x02}$ & $\texttt{0x04}$ & $\texttt{0x08}$ & $\texttt{0x10}$ & $\texttt{0x20}$ & $\texttt{0x40}$ & $\texttt{0x80}$ & $\texttt{0x1b}$ & $\texttt{0x36}$\\
		\end{tabular*}
	\end{table}
	\begin{lstlisting}[style=C, caption={rCon Array Declaration},captionpos=t]
static const u32 rCon[10] = {
	0x01000000, 0x02000000, 0x04000000, 0x08000000,
	0x10000000, 0x20000000, 0x40000000, 0x80000000,
	0x1b000000, 0x36000000
};
	\end{lstlisting}
\end{itemize}
\begin{algorithm}[H]
	\caption{Key Schedule (AES-128)}
	
	\KwIn{User key \( uk = (uk_0, \dots, uk_{15}) \) \( (uk_i \in \binaryfield^8) \)\tcp*{$uk\in\binaryfield^{128}$ is 16-byte}}
	\KwOut{round-keys \( \{rk_i\}_{i=0}^{43} \) \( (rk_i \in \binaryfield^{32}) \)\tcp*{$\set{rk_i}_{i=0}^{43}\in\binaryfield^{1408}$ is 176-byte}}
	\BlankLine
	\( rk_0 \leftarrow uk_0\parallel uk_1\parallel uk_2 \parallel uk_3 \)\;
	\( rk_1 \leftarrow uk_4\parallel uk_5\parallel uk_6 \parallel uk_7 \)\;
	\( rk_2 \leftarrow uk_8\parallel uk_9\parallel uk_{10} \parallel uk_{11} \)\;
	\( rk_3 \leftarrow uk_{12}\parallel uk_{13}\parallel uk_{14} \parallel uk_{15} \)\;
	\For{\( i = 4 \) \KwTo \( 43 \)}{
		\( t \leftarrow rk_{i-1} \)\;
		\If{\( i \bmod 4 = 0 \)}{
			\Comment{$\text{SubWord}\circ\text{RotWord}:\binaryfield^{32}\to\binaryfield^{32}$}
			\( t \leftarrow \text{RotWord}(t) \)\;
			\( t \leftarrow \text{SubWord}(t) \)\;
			\( t \leftarrow t \oplus (\texttt{rCon}_{i/4}\parallel\texttt{0x00}\parallel\texttt{0x00}\parallel\texttt{0x00}) \)\;
		}
		\( rk_i \leftarrow rk_{i-4} \oplus_{32} t \)\;
	}
\end{algorithm}
\vspace{24pt}
\begin{lstlisting}[style=C, caption={AES Key Expansion},captionpos=t]
void KeyExpansion(const u8* uKey, u32* rKey) {
	u32 temp;
	int i = 0;
	
	// Copy the input key to the first round key
	while (i < 4) {
		rKey[i] = (u32)uKey[4*i] << 0x18 | 
		(u32)uKey[4*i+1] << 0x10 | 
		(u32)uKey[4*i+2] << 0x08 | 
		(u32)uKey[4*i+3];
		i++;
	}
	
	i = 4;
	
	// Generate the remaining round keys
	while (i < 44) {
		temp = rKey[i-1];
		if (i % 4 == 0) {
			temp = SubWord(RotWord(temp)) ^ rCon[i/4-1];
		}
		rKey[i] = rKey[i-4] ^ temp;
		i++;
	}
}
\end{lstlisting}

\newpage
\subsection{AddRoundKey}
%/* 
%* +------------+------------+------------------------------------+
%* | Variable   | Type       | Description                        |
%* +------------+------------+------------------------------------+
%* | var1       | int        | Stores the count of items          |
%* | var2       | float      | Represents a percentage value      |
%* | ...        | ...        | ...                                |
%* +------------+------------+------------------------------------+
%*/
\begin{itemize}
	\item $\text{AddRoundKey}:\binaryfield^{128}\times\binaryfield^{128}\to\binaryfield^{128}$ is defined by \[
	\text{AddRoundKey}\left(\set{X_i}_{i=0}^{15},\set{rk_i}_{i=0}^{3}\right):=\set{X_i\oplus_{8} uk_i}_{i=0}^{15}.
	\] 
\end{itemize}
\begin{lstlisting}[style=C, caption={AES AddRoundKey},captionpos=t]
void AddRoundKey(u8* state, const u32* rKey) {
	for (int i = 0; i < AES_KEY_SIZE; i++) {
		// i =  0,  1,  2,  3 => wordIndex = 0
		// i =  4,  5,  6,  7 => wordIndex = 1
		// i =  8,  9, 10, 11 => wordIndex = 2
		// i = 12, 13, 14, 15 => wordIndex = 3 
		int wordIndex = i / 4;
		
		// i =  0,  1,  2,  3 => bytePosition = 0,  1,  2,  3
		// i =  4,  5,  6,  7 => bytePosition = 0,  1,  2,  3
		// i =  8,  9, 10, 11 => bytePosition = 0,  1,  2,  3
		// i = 12, 13, 14, 15 => bytePosition = 0,  1,  2,  3 
		int bytePosition = i % 4;
/* 
 * +-------+------------+--------------+----------------------+
 * | i     | wordIndex  | bytePosition | shiftedWord          |
 * +-------+------------+--------------+----------------------+
 * | 0-3   | 0          | 0            | rKey[0] >> 0x18      |
 * |       |            | 1            | rKey[0] >> 0x10      |
 * |       |            | 2            | rKey[0] >> 0x08      |
 * |       |            | 3            | rKey[0]              |
 * ------------------------------------------------------------
 * | 4-7   | 1          | 0            | rKey[1] >> 24        |
 * |       |            | 1            | rKey[1] >> 16        |
 * |       |            | 2            | rKey[1] >> 8         |
 * |       |            | 3            | rKey[1]              |
 * ------------------------------------------------------------
 * | ...   | ...        | ...          | ...                  |
 * ------------------------------------------------------------
 * | 15    | 3          | 3            | rKey[3]              |
 * +-------+------------+--------------+----------------------+
*/
		u32 shiftedWord =
			rKey[wordIndex] >> (8 * (3 - bytePosition));
		
		u8 keyByte = shiftedWord & 0xFF;
		state[i] ^= keyByte;
		
/* Extract the corresponding byte from the round key word */
// state[i] ^= (rKey[i / 4] >> (8 * (3 - (i % 4)))) & 0xFF;
	}
}
\end{lstlisting}

\subsection{SubBytes / InvSubBytes}
\begin{itemize}
	\item $\text{SubBytes}:\binaryfield^{128}\to\binaryfield^{128}$ is defined by \[
	\text{SubBytes}(\set{X_i}_{i=0}^{15})=\set{s(X_i)}_{i=0}^{15}.
	\]
	\item $\text{InvSubBytes}:\binaryfield^{128}\to\binaryfield^{128}$ is defined by \[
	\text{SubBytes}(\set{X_i}_{i=0}^{15})=\set{s^{-1}(X_i)}_{i=0}^{15}.
	\]
\end{itemize}

\newcommand{\headerrow}{%
	& 00 & 01 & 02 & 03 & 04 & 05 & 06 & 07 & 08 & 09 & 0a & 0b & 0c & 0d & 0e & 0f \\
}

\begin{longtable}{|c||*{16}{c|}}
	\caption{Substitution Box}\\
	\hline
	\headerrow \endhead % Header at the start of the table
	\hline\hline
	00 & \texttt{63} & \texttt{7c} & \texttt{77} & \texttt{7b} & \texttt{f2} & \texttt{6b} & \texttt{6f} & \texttt{c5} & \texttt{30} & \texttt{01} & \texttt{67} & \texttt{2b} & \texttt{fe} & \texttt{d7} & \texttt{ab} & \texttt{76} \\
	\hline
	10 & \texttt{ca} & \texttt{82} & ... & ... & ... & ... & ... & ... & ... & ... & ... & ... & ... & ... & ... & ... \\
	\hline
	30 & ... & ... & ... & ... & ... & ... & ... & ... & ... & ... & ... & ... & ... & ... & ... & ... \\
	\hline
	40 & ... & ... & ... & ... & ... & ... & ... & ... & ... & ... & ... & ... & ... & ... & ... & ... \\
	\hline
	50 & ... & ... & ... & ... & ... & ... & ... & ... & ... & ... & ... & ... & ... & ... & ... & ... \\
	\hline
	60 & ... & ... & ... & ... & ... & ... & ... & ... & ... & ... & ... & ... & ... & ... & ... & ... \\
	\hline
	70 & ... & ... & ... & ... & ... & ... & ... & ... & ... & ... & ... & ... & ... & ... & ... & ... \\
	\hline
	80 & ... & ... & ... & ... & ... & ... & ... & ... & ... & ... & ... & ... & ... & ... & ... & ... \\
	\hline
	90 & ... & ... & ... & ... & ... & ... & ... & ... & ... & ... & ... & ... & ... & ... & ... & ... \\
	\hline
	a0 & ... & ... & ... & ... & ... & ... & ... & ... & ... & ... & ... & ... & ... & ... & ... & ... \\
	\hline
	b0 & ... & ... & ... & ... & ... & ... & ... & ... & ... & ... & ... & ... & ... & ... & ... & ... \\
	\hline
	c0 & ... & ... & ... & ... & ... & ... & ... & ... & ... & ... & ... & ... & ... & ... & ... & ... \\
	\hline
	d0 & ... & ... & ... & ... & ... & ... & ... & ... & ... & ... & ... & ... & ... & \texttt{c1} & ... & ... \\
	\hline
	e0 & ... & ... & ... & ... & ... & ... & ... & ... & ... & ... & ... & ... & ... & ... & \texttt{28} & ... \\
	\hline
	f0 & ... & ... & ... & ... & ... & ... & ... & ... & ... & ... & ... & ... & ... & ... & ... & \texttt{16} \\
	\hline
	% Continue filling in the rest of the rows in the same pattern
\end{longtable}

\begin{lstlisting}[style=C, caption={Byte Substitution},captionpos=t]
void SubBytes(u8* state) {
	for (int i = 0; i < AES_KEY_SIZE; i++) {
		state[i] = s_box[state[i]];
	}
}
\end{lstlisting}

\begin{lstlisting}[style=C, caption={Inverse Byte Substitution},captionpos=t]
void SubBytes(u8* state) {
	for (int i = 0; i < AES_KEY_SIZE; i++) {
		state[i] = inv_s_box[state[i]];
	}
}
\end{lstlisting}

\newpage
\subsection{ShiftRows / InvShiftRows}
\begin{itemize}
	\item $\text{ShiftRows}:\binaryfield^{128}\to\binaryfield^{128}$ is defined by
	\begin{center}
	\begin{minipage}{.4\textwidth}\centering
		\begin{tabular}{|c|c|c|c|}
			\hline
			\cellcolor{red!20}$X_0$ & \cellcolor{red!20}$X_4$ & \cellcolor{red!20}$X_8$ & \cellcolor{red!20}$X_{12}$ \\ \hline
			\cellcolor{green!20}$X_1$ & \cellcolor{blue!20}$X_5$ & \cellcolor{blue!20}$X_9$ & \cellcolor{blue!20}$X_{13}$ \\ \hline
			\cellcolor{green!20}$X_2$ & \cellcolor{green!20}$X_6$ & \cellcolor{blue!20}$X_{10}$ & \cellcolor{blue!20}$X_{14}$ \\ \hline
			\cellcolor{green!20}$X_3$ & \cellcolor{green!20}$X_7$ & \cellcolor{green!20}$X_{11}$ & \cellcolor{blue!20}$X_{15}$ \\ \hline
		\end{tabular}
	\end{minipage}$\implies$\begin{minipage}{.4\textwidth}\centering
	\begin{tabular}{|c|c|c|c|}
		\hline
		\cellcolor{red!20}$X_0$ & \cellcolor{red!20}$X_4$ & \cellcolor{red!20}$X_8$ & \cellcolor{red!20}$X_{12}$ \\ \hline
		\cellcolor{blue!20}$X_5$ & \cellcolor{blue!20}$X_9$ & \cellcolor{blue!20}$X_{13}$ & \cellcolor{green!20}$X_1$ \\ \hline
		\cellcolor{blue!20}$X_{10}$ & \cellcolor{blue!20}$X_{14}$ & \cellcolor{green!20}$X_2$ & \cellcolor{green!20}$X_6$ \\ \hline
		\cellcolor{blue!20}$X_{15}$ & \cellcolor{green!20}$X_3$ & \cellcolor{green!20}$X_7$ & \cellcolor{green!20}$X_{11}$ \\ \hline
	\end{tabular}
\end{minipage}
	\end{center}
	\item $\text{InvShiftRows}:\binaryfield^{128}\to\binaryfield^{128}$ is defined by
	\begin{center}
	\begin{minipage}{.4\textwidth}\centering
	\begin{tabular}{|c|c|c|c|}
		\hline
		\cellcolor{red!20}$X_0$ & \cellcolor{red!20}$X_4$ & \cellcolor{red!20}$X_8$ & \cellcolor{red!20}$X_{12}$ \\ \hline
		\cellcolor{green!20}$X_1$ & \cellcolor{green!20}$X_5$ & \cellcolor{green!20}$X_9$ & \cellcolor{blue!20}$X_{13}$ \\ \hline
		\cellcolor{green!20}$X_2$ & \cellcolor{green!20}$X_6$ & \cellcolor{blue!20}$X_{10}$ & \cellcolor{blue!20}$X_{14}$ \\ \hline
		\cellcolor{green!20}$X_3$ & \cellcolor{blue!20}$X_7$ & \cellcolor{blue!20}$X_{11}$ & \cellcolor{blue!20}$X_{15}$ \\ \hline
	\end{tabular}
\end{minipage}$\implies$\begin{minipage}{.4\textwidth}\centering
	\begin{tabular}{|c|c|c|c|}
		\hline
		\cellcolor{red!20}$X_0$ & \cellcolor{red!20}$X_4$ & \cellcolor{red!20}$X_8$ & \cellcolor{red!20}$X_{12}$ \\ \hline
		\cellcolor{blue!20}$X_{13}$ & \cellcolor{green!20}$X_1$ & \cellcolor{green!20}$X_{5}$ & \cellcolor{green!20}$X_9$ \\ \hline
		\cellcolor{blue!20}$X_{10}$ & \cellcolor{blue!20}$X_{14}$ & \cellcolor{green!20}$X_2$ & \cellcolor{green!20}$X_6$ \\ \hline
		\cellcolor{blue!20}$X_{7}$ & \cellcolor{blue!20}$X_{11}$ & \cellcolor{blue!20}$X_{15}$ & \cellcolor{green!20}$X_{3}$ \\ \hline
	\end{tabular}
\end{minipage}
	\end{center}
\end{itemize}
\vspace{24pt}
\begin{lstlisting}[style=C, caption={ShiftRows},captionpos=t]
void ShiftRows(u8* state) {
	u8 temp;
	
	// Row 1: shift left by 1
	temp = state[1];
	state[1] = state[5];
	state[5] = state[9];
	state[9] = state[13];
	state[13] = temp;
	
	// Row 2: shift left by 2
	temp = state[2];
	state[2] = state[10];
	state[10] = temp;
	temp = state[6];
	state[6] = state[14];
	state[14] = temp;
	
	// Row 3: shift left by 3 (or right by 1)
	temp = state[15];
	state[15] = state[11];
	state[11] = state[7];
	state[7] = state[3];
	state[3] = temp;
}
\end{lstlisting}
\newpage
\begin{lstlisting}[style=C, caption={Inverse ShiftRows},captionpos=t]
void InvShiftRows(u8* state) {
	u8 temp;
	
	// Row 1: shift left by 3 (or right by 1)
	temp = state[13];
	state[13] = state[9];
	state[9] = state[5];
	state[5] = state[1];
	state[1] = temp;
	
	// Row 2: shift left by 2
	temp = state[2];
	state[2] = state[10];
	state[10] = temp;
	temp = state[6];
	state[6] = state[14];
	state[14] = temp;
	
	// Row 3: shift left by 1
	temp = state[3];
	state[3] = state[7];
	state[7] = state[11];
	state[11] = state[15];
	state[15] = temp;
}
\end{lstlisting}

\newpage
\subsection{MixColumns / InvMixColumns}
\begin{itemize}
	\item Multiplication in the finite filed $\text{GF}(2^8)$. \[
	\text{MUL}_{GF256}:\binaryfield^{8}\times\binaryfield^8\to\binaryfield^8.
	\] Here, \[
	\binaryfield^{8}\simeq GF(2^8)=\F_{2^8}:=\F_2[z]/(z^8+z^4+z^3+z+1)=\set{a_7z^7+\cdots+a_1z+a_0:a_i\in\F_2}.
	\]
	Note that \[
	a(z)\times b(z):= a(z)\times b(z) \bmod (z^8+z^4+z^3+z+1)
	\]
	\vspace{12pt}
	\begin{note}
		Given two polynomials \( a(x) \) and \( b(x) \) in \( GF(2^8) \):
		\begin{align*}
			a(x) &= a_7x^7 + a_6x^6 + a_5x^5 + a_4x^4 + a_3x^3 + a_2x^2 + a_1x + a_0, \\
			b(x) &= b_7x^7 + b_6x^6 + b_5x^5 + b_4x^4 + b_3x^3 + b_2x^2 + b_1x + b_0.
		\end{align*}
		The algorithm performs polynomial multiplication in the finite field \( GF(2^8) \). It uses a shift-and-add method, with an additional reduction step modulo an irreducible polynomial \( m(x) = x^8 + x^4 + x^3 + x + 1 \).
		\begin{enumerate}
			\item Initialization: Set \( p(x) = 0 \) to initialize the product polynomial.
			\item Iterate over each bit of \( b(x) \), from LSB to MSB.
			\begin{enumerate}[(i)]
				\item If the current bit \( b_i \) of \( b(x) \) is 1, update \( p(x) \) as \( p(x) \oplus a(x) \). In \( GF(2^8) \), addition is equivalent to the XOR operation: \[
				p(x)=p(x)\oplus a(x).
				\]
				\item Shift \( a(x) \) left by 1 (multiply by \( x \)), increasing its degree by 1: \[
				a(x) = a(x)\cdot x.
				\]
				\item If the coefficient of \( x^8 \) in \( a(x) \) is 1, reduce \( a(x) \) by \( m(x) \) to keep the degree under 8: \[
				a(x)=a(x)\oplus m(x).
				\]
				\item Shift \( b(x) \) right by 1 (divide by \( x \)) for the next iteration: \[
				b(x) = b(x)\ /\ x.
				\] 
			\end{enumerate}
			\item After all bits of \( b(x) \) are processed, \( p(x) \) be the product of \( a(x) \) and \( b(x) \) modulo \( m(x) \).
		\end{enumerate}
	\end{note}

	\newpage
	
	\newpage
	\begin{note}[\bf Modular Reduction in $GF(2^8)$ using XOR]	
	In the context of multiplication in the binary finite field $GF(2^8)$, modular reduction ensures that results of operations remain within the field. The use of XOR for modular reduction is due to the properties of polynomial arithmetic over GF(2) and the representation of elements in $GF(2^8)$.
	\begin{itemize}
		\item \textbf{Polynomial Representation in $GF(2^8)$:}
		\begin{enumerate}
			\item \textbf{Elements as Polynomials}: Each element in $GF(2^8)$ can be represented as a polynomial of degree less than 8, where each coefficient is either 0 or 1, \ie, \[
			GF(2^8)=\F_2[x]/(x^8+x^4+x^3+x+1)=\set{a_7x^7+\cdots+a_1x+a_0:a_i\in\F_2}.
			\] This corresponds to an 8-bit binary number, with each bit representing a coefficient of the polynomial, \ie, \[
			a_7x^7+\cdots+a_1x+a_0\iff(a_7\dots a_1a_0)_2.
			\]
			\item \textbf{Binary Operations}: In $GF(2)$, addition and subtraction are equivalent to the XOR operation, since \(1 + 1 = 0\) in this field, the same as \(1 \oplus 1\).
		\end{enumerate}
		\vspace{12pt}
		\item \textbf{Modular Reduction with an Irreducible Polynomial}
		\begin{enumerate}
			\item \textbf{Irreducible Polynomial}: In $GF(2^8)$, an irreducible polynomial of degree 8, typically \( p(x) = x^8 + x^4 + x^3 + x + 1 \) (represented as \texttt{0x11b} in binary), is used for modular reduction.
			
			\item \textbf{Modular Reduction Process}: After multiplying two polynomials, if the resulting polynomial's degree is 8 or higher, it must be reduced modulo the irreducible polynomial to ensure the result remains a polynomial of degree less than 8, thus staying within $GF(2^8)$.
			
			\item \textbf{XOR for Reduction}: XOR is used for modular reduction in $GF(2^8)$ because polynomial subtraction in GF(2) is performed by XORing coefficients. 
		\end{enumerate}
		\item 
				Given two elements in $GF(2^8)$, \( a(x) \) and \( b(x) \), their product is \( c(x) = a(x) \cdot b(x) \). If \( \deg(c(x)) \geq 8 \), then \( c(x) \) must be reduced modulo the irreducible polynomial \( p(x) \). This is achieved by XORing the coefficients of \( c(x) \) and \( p(x) \):
				
				\[
				c(x) = a(x) \cdot b(x) \mod p(x)
				\]
				
				If \( c(x) \) has a term \( x^8 \) or higher, we subtract \( p(x) \) from \( c(x) \) to reduce its degree. In GF(2), subtraction is equivalent to addition, performed by XORing coefficients:
				
				\[
				c'(x) = c(x) \oplus p(x)
				\]
				
				This operation effectively eliminates the term \( x^8 \) (or higher) in \( c(x) \), ensuring that the result remains within $GF(2^8)$.
					Consider the product of two polynomials \( a(x) \) and \( b(x) \) in $GF(2^8)$:
					
					\[
					a(x) = x^6 + x^4 + x^2 + x + 1 \quad \text{and} \quad b(x) = x^7 + x + 1
					\]
					
					The product \( c(x) = a(x) \cdot b(x) \) might yield a polynomial of degree 8 or higher. To reduce \( c(x) \) modulo \( p(x) = x^8 + x^4 + x^3 + x + 1 \), we perform XOR between the coefficients of \( c(x) \) and \( p(x) \), ensuring the result stays within $GF(2^8)$.
	\end{itemize}
	
%		
%		
%		
%		\subsection*
%		

%		
%		\subsection*{Detailed Explanation}
%		

%		
%		\subsection*{Example}
%		

\end{note}

	\newpage
	\begin{lstlisting}[style=C, caption={Multiplication in GF($2^8$)},captionpos=t]
u8 MUL_GF256(u8 a, u8 b) {
	u8 res = 0;
	// Mask for detecting the MSB (0x80 = 0b10000000)
	u8 MSB_mask = 0x80;
	u8 MSB;
	/*
	 * The reduction polynomial
	 * (x^8 + x^4 + x^3 + x + 1) = 0b100011011
	 * for AES, represented in hexadecimal
	*/
	u8 modulo = 0x1B;
	
	for (int i = 0; i < 8; i++) {
		// Add a to result if LSB(b)=1
		if (b & 1)
			res ^= a;
		
		MSB = a & MSB_mask; // Store the MSB of a
		a <<= 1; // Multiplying it by x effectively
		
		// Reduce the result modulo the reduction polynomial
		if (MSB)
			a ^= modulo;
		
		b >>= 1; // Moving to the next bit
	}
	
	return res;
}

#define MUL_GF256(a, b) ({ \
	u8 res = 0; \
	u8 MSB_mask = 0x80; \
	u8 MSB; \
	u8 modulo = 0x1B; \
	u8 temp_a = (a); \
	u8 temp_b = (b); \
	for (int i = 0; i < 8; i++) { \
		if (temp_b & 1) \
		res ^= temp_a; \
		MSB = temp_a & MSB_mask; \
		temp_a <<= 1; \
		if (MSB) \
		temp_a ^= modulo; \
		temp_b >>= 1; \
	} \
	res; \
})
	\end{lstlisting}
	\newpage
	\item $\text{MixColumns}:\binaryfield^{128}\to\binaryfield^{128}$ is defined by \[
	\text{MixColumns}\left(\begin{pmatrix}
		X_0 & X_4 & X_8 & X_{12} \\
		X_1 & X_5 & X_9 & X_{13} \\
		X_2 & X_6 & X_{10} & X_{14} \\
		X_3 & X_7 & X_{11} & X_{15}
	\end{pmatrix}\right):=\begin{pmatrix}
		\texttt{0x02} & \texttt{0x03} & \texttt{0x01} & \texttt{0x01}\\
		\texttt{0x01} & \texttt{0x02} & \texttt{0x03} & \texttt{0x01}\\
		\texttt{0x01} & \texttt{0x01} & \texttt{0x02} & \texttt{0x03}\\
		\texttt{0x03} & \texttt{0x01} & \texttt{0x01} & \texttt{0x02}
	\end{pmatrix}\begin{pmatrix}
		X_0 & X_4 & X_8 & X_{12} \\
		X_1 & X_5 & X_9 & X_{13} \\
		X_2 & X_6 & X_{10} & X_{14} \\
		X_3 & X_7 & X_{11} & X_{15}
		\end{pmatrix}.
	\]
	\item $\text{InvMixColums}:\binaryfield^{128}\to\binaryfield^{128}$ is defined by \[
	\text{MixColums}\left(\begin{pmatrix}
		X_0 & X_4 & X_8 & X_{12} \\
		X_1 & X_5 & X_9 & X_{13} \\
		X_2 & X_6 & X_{10} & X_{14} \\
		X_3 & X_7 & X_{11} & X_{15}
	\end{pmatrix}\right):=\begin{pmatrix}
		\texttt{0x0e} & \texttt{0x0b} & \texttt{0x0d} & \texttt{0x09}\\
		\texttt{0x09} & \texttt{0x0e} & \texttt{0x0b} & \texttt{0x0d}\\
		\texttt{0x0d} & \texttt{0x09} & \texttt{0x0e} & \texttt{0x0b}\\
		\texttt{0x0b} & \texttt{0x0d} & \texttt{0x09} & \texttt{0x0e}
	\end{pmatrix}\begin{pmatrix}
		X_0 & X_4 & X_8 & X_{12} \\
		X_1 & X_5 & X_9 & X_{13} \\
		X_2 & X_6 & X_{10} & X_{14} \\
		X_3 & X_7 & X_{11} & X_{15}
	\end{pmatrix}.
	\]
\end{itemize}
\begin{lstlisting}[style=C, caption={MixColumns},captionpos=t]
void MixColumns(u8* state) {
	u8 temp[4];
	// Multiply and add the elements in the column
	// by the fixed polynomial
	for (int i = 0; i < 4; i++) { 
		temp[0] =
			MUL_GF256(0x02, state[i * 4]) ^
			MUL_GF256(0x03, state[i * 4 + 1]) ^
			state[i * 4 + 2] ^
			state[i * 4 + 3]; 
		
		temp[1] =
			state[i * 4] ^
			MUL_GF256(0x02, state[i * 4 + 1]) ^
			MUL_GF256(0x03, state[i * 4 + 2]) ^
			state[i * 4 + 3];
		
		temp[2] =
			state[i * 4] ^
			state[i * 4 + 1] ^
			MUL_GF256(0x02, state[i * 4 + 2]) ^
			MUL_GF256(0x03, state[i * 4 + 3]);
		
		temp[3] =
			MUL_GF256(0x03, state[i * 4]) ^
			state[i * 4 + 1] ^
			state[i * 4 + 2] ^
			MUL_GF256(0x02, state[i * 4 + 3]);
		
		// Copy the mixed column back to the state
		for (int j = 0; j < 4; j++)
			state[i * 4 + j] = temp[j];
	}
}
\end{lstlisting}
\newpage
\begin{lstlisting}[style=C, caption={Inverse MixColumns},captionpos=t]
void InvMixColumns(u8* state) {
	u8 temp[4];
	
	for (int i = 0; i < 4; i++) { 
		temp[0] =
			MUL_GF256(0x0e, state[i * 4]) ^
			MUL_GF256(0x0b, state[i * 4 + 1]) ^
			MUL_GF256(0x0d, state[i * 4 + 2]) ^
			MUL_GF256(0x09, state[i * 4 + 3]); 
		
		temp[1] =
			MUL_GF256(0x09, state[i * 4]) ^
			MUL_GF256(0x0e, state[i * 4 + 1]) ^
			MUL_GF256(0x0b, state[i * 4 + 2]) ^
			MUL_GF256(0x0d, state[i * 4 + 3]);
		
		temp[2] =
			MUL_GF256(0x0d, state[i * 4]) ^
			MUL_GF256(0x09, state[i * 4 + 1]) ^
			MUL_GF256(0x0e, state[i * 4 + 2]) ^
			MUL_GF256(0x0b, state[i * 4 + 3]);
		
		temp[3] =
			MUL_GF256(0x0b, state[i * 4]) ^
			MUL_GF256(0x0d, state[i * 4 + 1]) ^
			MUL_GF256(0x09, state[i * 4 + 2]) ^
			MUL_GF256(0x0e, state[i * 4 + 3]);
		
		for (int j = 0; j < 4; j++)
			state[i * 4 + j] = temp[j];
	}
}
\end{lstlisting}