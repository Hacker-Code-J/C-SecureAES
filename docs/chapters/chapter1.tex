\chapter{Block Cipher AES-128}

\section{Overview of Advanced Encryption Standard}

\begin{itemize}
	\item $\text{KeyExpansion}:\binaryfield^{128}\to\binaryfield^{1408}$.
	\item $\text{AddRoundKey}:\binaryfield^{128}\times\binaryfield^{128}\to\binaryfield^{128}$.
	\item $\text{SubBytes}:\binaryfield^{128}\to\binaryfield^{128}$.
	\item $\text{ShiftRows}:\binaryfield^{128}\to\binaryfield^{128}$.
	\item $\text{MixColumns}:\binaryfield^{128}\to\binaryfield^{128}$.
\end{itemize}

\begin{algorithm}[H]
	\caption{Encryption of AES-128}
	
	\KwIn{block $\mathsf{src} \in \binaryfield^{128}$, round-keys $\{rk_i\}_{i=0}^{11}$ ($rk_i \in \binaryfield^{128}$)}
	\KwOut{block $\mathsf{dst} \in \binaryfield^{128}$}
	\BlankLine
	$t \leftarrow \mathsf{src}$\;
	$t \leftarrow \text{AddRoundKey}(t, rk_0)$\;
	\For{$i \leftarrow 1$ \KwTo $9$}{
		$t \leftarrow \text{SubBytes}(t)$\;
		$t \leftarrow \text{ShiftRows}(t)$\;
		$t \leftarrow \text{MixColumns}(t)$\;
		$t \leftarrow \text{AddRoundKey}(t, rk_i)$\;
	}
	$t \leftarrow \text{SubBytes}(t)$\;
	$t \leftarrow \text{ShiftRows}(t)$\;
	$t \leftarrow \text{AddRoundKey}(t, rk_{10})$\;
	$\mathsf{dst} \leftarrow t$\;
	\Return{$\mathsf{dst}$}\;
\end{algorithm}

\newpage
\section{Functions and Constants used in AES}
\subsection{Key Expansion}
\begin{itemize}
	\item \hl{$\text{RotWord}:\binaryfield^{32}\to\binaryfield^{32}$} is defined by \[
	\text{RotWord}\left(X_0\parallel X_1\parallel X_2\parallel X_3\right):=X_1\parallel X_2\parallel X_3\parallel X_0\quad\text{for}\quad X_i\in\binaryfield^8.
	\]
	\begin{lstlisting}[style=C, caption={RotWord rotates the input word left by one byte},captionpos=t]
u32 RotWord(u32 word) {
	return (word << 0x08) | (word >> 0x18);
}
	\end{lstlisting}
	\item \hl{$\text{SubWord}:\binaryfield^{32}\to\binaryfield^{32}$} is defined by \[
	\text{SubWord}(X_0\parallel X_1\parallel X_2\parallel X_3):=s(X_0)\parallel s(X_1)\parallel s(X_2)\parallel s(X_3)\quad\text{for}\quad X_i\in\binaryfield^8.
	\] Here, $s:{\binaryfield^8}\to{\binaryfield^8}$ is the \hyperlink{s-box}{S-box}.
	\begin{lstlisting}[style=C, caption={SubWord applies the S-box to each byte of the input word},captionpos=t]
u32 SubWord(u32 word) {
	return (u32)s_box[word >> 0x18] << 0x18 | 
		(u32)s_box[(word >> 0x10) & 0xFF] << 0x10 | 
		(u32)s_box[(word >> 0x08) & 0xFF] << 0x08 | 
		(u32)s_box[word & 0xFF];
}
	\end{lstlisting}
	\item \hl{Round Constant $\texttt{rCon}$}:
	
	The constant $\texttt{rCon}_i\in\F_{2^8}$ used in generating the $i$-th round key corresponds to the value of $x^{i-1}$ in the binary finite field $\F_{2^8}$ and is as follows:
	\begin{table}[h!]\centering\renewcommand{\arraystretch}{1.5} % Increase row height by 1.5 times
		\begin{tabular*}{\textwidth}{@{\extracolsep{\fill}}c||cccccccccc}
			$i$ & 1 & 2 & 3 & 4 & 5 & 6 & 7 & 8 & 9 & 10\\
			\hline
			$\texttt{Rcon}_i$ & $\texttt{0x01}$ & $\texttt{0x02}$ & $\texttt{0x04}$ & $\texttt{0x08}$ & $\texttt{0x10}$ & $\texttt{0x20}$ & $\texttt{0x40}$ & $\texttt{0x80}$ & $\texttt{0x1b}$ & $\texttt{0x36}$\\
		\end{tabular*}
	\end{table}
	\begin{lstlisting}[style=C, caption={rCon Array Declaration},captionpos=t]
static const u32 rCon[10] = {
	0x01000000, 0x02000000, 0x04000000, 0x08000000,
	0x10000000, 0x20000000, 0x40000000, 0x80000000,
	0x1b000000, 0x36000000
};
	\end{lstlisting}
\end{itemize}
\begin{algorithm}[H]
	\caption{Key Schedule (AES-128)}
	
	\KwIn{User key \( uk = (uk_0, \dots, uk_{15}) \) \( (uk_i \in \binaryfield^8) \)\tcp*{$uk\in\binaryfield^{128}$ is 16-byte}}
	\KwOut{round-keys \( \{rk_i\}_{i=0}^{43} \) \( (rk_i \in \binaryfield^{32}) \)\tcp*{$\set{rk_i}_{i=0}^{43}\in\binaryfield^{1408}$ is 176-byte}}
	\BlankLine
	\( rk_0 \leftarrow uk_0\parallel uk_1\parallel uk_2 \parallel uk_3 \)\;
	\( rk_1 \leftarrow uk_4\parallel uk_5\parallel uk_6 \parallel uk_7 \)\;
	\( rk_2 \leftarrow uk_8\parallel uk_9\parallel uk_{10} \parallel uk_{11} \)\;
	\( rk_3 \leftarrow uk_{12}\parallel uk_{13}\parallel uk_{14} \parallel uk_{15} \)\;
	\For{\( i = 4 \) \KwTo \( 43 \)}{
		\( t \leftarrow rk_{i-1} \)\;
		\If{\( i \bmod 4 = 0 \)}{
			\Comment{$\text{SubWord}\circ\text{RotWord}:\binaryfield^{32}\to\binaryfield^{32}$}
			\( t \leftarrow \text{RotWord}(t) \)\;
			\( t \leftarrow \text{SubWord}(t) \)\;
			\( t \leftarrow t \oplus (\texttt{rCon}_{i/4}\parallel\texttt{0x00}\parallel\texttt{0x00}\parallel\texttt{0x00}) \)\;
		}
		\( rk_i \leftarrow rk_{i-4} \oplus_{32} t \)\;
	}
\end{algorithm}
\vspace{24pt}
\begin{lstlisting}[style=C, caption={AES Key Expansion},captionpos=t]
void KeyExpansion(const u8* uKey, u32* rKey) {
	u32 temp;
	int i = 0;
	
	// Copy the input key to the first round key
	while (i < 4) {
		rKey[i] = (u32)uKey[4*i] << 0x18 | 
		(u32)uKey[4*i+1] << 0x10 | 
		(u32)uKey[4*i+2] << 0x08 | 
		(u32)uKey[4*i+3];
		i++;
	}
	
	i = 4;
	
	// Generate the remaining round keys
	while (i < 44) {
		temp = rKey[i-1];
		if (i % 4 == 0) {
			temp = SubWord(RotWord(temp)) ^ rCon[i/4-1];
		}
		rKey[i] = rKey[i-4] ^ temp;
		i++;
	}
}
\end{lstlisting}

\newpage
\subsection{AddRoundKey}
%/* 
%* +------------+------------+------------------------------------+
%* | Variable   | Type       | Description                        |
%* +------------+------------+------------------------------------+
%* | var1       | int        | Stores the count of items          |
%* | var2       | float      | Represents a percentage value      |
%* | ...        | ...        | ...                                |
%* +------------+------------+------------------------------------+
%*/
\begin{itemize}
	\item $\text{AddRoundKey}:\binaryfield^{128}\times\binaryfield^{128}\to\binaryfield^{128}$ is defined by \[
	\text{AddRoundKey}\left(\set{X_i}_{i=0}^{15},\set{rk_i}_{i=0}^{3}\right):=\set{X_i\oplus_{8} uk_i}_{i=0}^{15}.
	\] 
\end{itemize}
\begin{lstlisting}[style=C, caption={AES AddRoundKey},captionpos=t]
void AddRoundKey(u8* state, const u32* rKey) {
	for (int i = 0; i < AES_KEY_SIZE; i++) {
		// i =  0,  1,  2,  3 => wordIndex = 0
		// i =  4,  5,  6,  7 => wordIndex = 1
		// i =  8,  9, 10, 11 => wordIndex = 2
		// i = 12, 13, 14, 15 => wordIndex = 3 
		int wordIndex = i / 4;
		
		// i =  0,  1,  2,  3 => bytePosition = 0,  1,  2,  3
		// i =  4,  5,  6,  7 => bytePosition = 0,  1,  2,  3
		// i =  8,  9, 10, 11 => bytePosition = 0,  1,  2,  3
		// i = 12, 13, 14, 15 => bytePosition = 0,  1,  2,  3 
		int bytePosition = i % 4;
/* 
 * +-------+------------+--------------+----------------------+
 * | i     | wordIndex  | bytePosition | shiftedWord          |
 * +-------+------------+--------------+----------------------+
 * | 0-3   | 0          | 0            | rKey[0] >> 0x18      |
 * |       |            | 1            | rKey[0] >> 0x10      |
 * |       |            | 2            | rKey[0] >> 0x08      |
 * |       |            | 3            | rKey[0]              |
 * ------------------------------------------------------------
 * | 4-7   | 1          | 0            | rKey[1] >> 24        |
 * |       |            | 1            | rKey[1] >> 16        |
 * |       |            | 2            | rKey[1] >> 8         |
 * |       |            | 3            | rKey[1]              |
 * ------------------------------------------------------------
 * | ...   | ...        | ...          | ...                  |
 * ------------------------------------------------------------
 * | 15    | 3          | 3            | rKey[3]              |
 * +-------+------------+--------------+----------------------+
*/
		u32 shiftedWord =
			rKey[wordIndex] >> (8 * (3 - bytePosition));
		
		u8 keyByte = shiftedWord & 0xFF;
		state[i] ^= keyByte;
		
/* Extract the corresponding byte from the round key word */
// state[i] ^= (rKey[i / 4] >> (8 * (3 - (i % 4)))) & 0xFF;
	}
}
\end{lstlisting}

\subsection{SubBytes}
\begin{itemize}
	\item $\text{SubBytes}:\binaryfield^{128}\to\binaryfield^{128}$ is defined by \[
	\text{SubBytes}(\set{X_i}_{i=0}^{15})=\set{s(X_i)}_{i=0}^{15}.
	\]
\end{itemize}

\newcommand{\headerrow}{%
	& 00 & 01 & 02 & 03 & 04 & 05 & 06 & 07 & 08 & 09 & 0a & 0b & 0c & 0d & 0e & 0f \\
}

\begin{longtable}{|c||*{16}{c|}}
	\caption{Substitution Box}\\
	\hline
	\headerrow \endhead % Header at the start of the table
	\hline\hline
	00 & \texttt{63} & \texttt{7c} & \texttt{77} & \texttt{7b} & \texttt{f2} & \texttt{6b} & \texttt{6f} & \texttt{c5} & \texttt{30} & \texttt{01} & \texttt{67} & \texttt{2b} & \texttt{fe} & \texttt{d7} & \texttt{ab} & \texttt{76} \\
	\hline
	10 & \texttt{ca} & \texttt{82} & ... & ... & ... & ... & ... & ... & ... & ... & ... & ... & ... & ... & ... & ... \\
	\hline
	30 & ... & ... & ... & ... & ... & ... & ... & ... & ... & ... & ... & ... & ... & ... & ... & ... \\
	\hline
	40 & ... & ... & ... & ... & ... & ... & ... & ... & ... & ... & ... & ... & ... & ... & ... & ... \\
	\hline
	50 & ... & ... & ... & ... & ... & ... & ... & ... & ... & ... & ... & ... & ... & ... & ... & ... \\
	\hline
	60 & ... & ... & ... & ... & ... & ... & ... & ... & ... & ... & ... & ... & ... & ... & ... & ... \\
	\hline
	70 & ... & ... & ... & ... & ... & ... & ... & ... & ... & ... & ... & ... & ... & ... & ... & ... \\
	\hline
	80 & ... & ... & ... & ... & ... & ... & ... & ... & ... & ... & ... & ... & ... & ... & ... & ... \\
	\hline
	90 & ... & ... & ... & ... & ... & ... & ... & ... & ... & ... & ... & ... & ... & ... & ... & ... \\
	\hline
	a0 & ... & ... & ... & ... & ... & ... & ... & ... & ... & ... & ... & ... & ... & ... & ... & ... \\
	\hline
	b0 & ... & ... & ... & ... & ... & ... & ... & ... & ... & ... & ... & ... & ... & ... & ... & ... \\
	\hline
	c0 & ... & ... & ... & ... & ... & ... & ... & ... & ... & ... & ... & ... & ... & ... & ... & ... \\
	\hline
	d0 & ... & ... & ... & ... & ... & ... & ... & ... & ... & ... & ... & ... & ... & \texttt{c1} & ... & ... \\
	\hline
	e0 & ... & ... & ... & ... & ... & ... & ... & ... & ... & ... & ... & ... & ... & ... & \texttt{28} & ... \\
	\hline
	f0 & ... & ... & ... & ... & ... & ... & ... & ... & ... & ... & ... & ... & ... & ... & ... & \texttt{16} \\
	\hline
	% Continue filling in the rest of the rows in the same pattern
\end{longtable}

\begin{lstlisting}[style=C, caption={Byte Substitution},captionpos=t]
void SubBytes(u8* state) {
	for (int i = 0; i < AES_KEY_SIZE; i++) {
		state[i] = s_box[state[i]];
	}
}
\end{lstlisting}

\subsection{ShiftRows}
\begin{itemize}
	\item $\text{ShiftRows}:\binaryfield^{128}\to\binaryfield^{128}$ is defined by
	\begin{center}
	\begin{minipage}{.4\textwidth}\centering
		\begin{tabular}{|c|c|c|c|}
			\hline
			\cellcolor{red!20}$X_0$ & \cellcolor{red!20}$X_4$ & \cellcolor{red!20}$X_8$ & \cellcolor{red!20}$X_{12}$ \\ \hline
			\cellcolor{green!20}$X_1$ & \cellcolor{blue!20}$X_5$ & \cellcolor{blue!20}$X_9$ & \cellcolor{blue!20}$X_{13}$ \\ \hline
			\cellcolor{green!20}$X_2$ & \cellcolor{green!20}$X_6$ & \cellcolor{blue!20}$X_{10}$ & \cellcolor{blue!20}$X_{14}$ \\ \hline
			\cellcolor{green!20}$X_3$ & \cellcolor{green!20}$X_7$ & \cellcolor{green!20}$X_{11}$ & \cellcolor{blue!20}$X_{15}$ \\ \hline
		\end{tabular}
	\end{minipage}$\implies$\begin{minipage}{.4\textwidth}\centering
	\begin{tabular}{|c|c|c|c|}
		\hline
		\cellcolor{red!20}$X_0$ & \cellcolor{red!20}$X_4$ & \cellcolor{red!20}$X_8$ & \cellcolor{red!20}$X_{12}$ \\ \hline
		\cellcolor{blue!20}$X_5$ & \cellcolor{blue!20}$X_9$ & \cellcolor{blue!20}$X_{13}$ & \cellcolor{green!20}$X_1$ \\ \hline
		\cellcolor{blue!20}$X_{10}$ & \cellcolor{blue!20}$X_{14}$ & \cellcolor{green!20}$X_2$ & \cellcolor{green!20}$X_6$ \\ \hline
		\cellcolor{blue!20}$X_{15}$ & \cellcolor{green!20}$X_3$ & \cellcolor{green!20}$X_7$ & \cellcolor{green!20}$X_{11}$ \\ \hline
	\end{tabular}
\end{minipage}
	\end{center}
\end{itemize}

\newpage
\begin{lstlisting}[style=C, caption={ShiftRows},captionpos=t]
void ShiftRows(u8* state) {
	u8 temp;
	
	// Row 1: shift left by 1
	temp = state[1];
	state[1] = state[5];
	state[5] = state[9];
	state[9] = state[13];
	state[13] = temp;
	
	// Row 2: shift left by 2
	temp = state[2];
	state[2] = state[10];
	state[10] = temp;
	temp = state[6];
	state[6] = state[14];
	state[14] = temp;
	
	// Row 3: shift left by 3 (or right by 1)
	temp = state[15];
	state[15] = state[11];
	state[11] = state[7];
	state[7] = state[3];
	state[3] = temp;
}
\end{lstlisting}


\subsection{MixColums}

\newpage
\section{Code Structure}
\begin{enumerate}
	\item Rcon Array Declaration
	\item Function Definition
	\item Variable Declarations and Initial Checks
	\item Key Expansion Logic
\end{enumerate}

\section{Detailed Analysis}
\subsection{Rcon Array Declaration}


\subsection{Function Definition}
\begin{lstlisting}[language=C]
	int AES_set_encrypt_key(const unsigned char *userKey, const int bits, AES_KEY *key);
\end{lstlisting}

\subsection{Variable Declarations and Initial Checks}
\begin{lstlisting}[language=C]
	u32 *rk;
	int i = 0;
	u32 temp;
	if (!userKey || !key)
	return -1;
	if (bits != 128 && bits != 192 && bits != 256)
	return -2;
\end{lstlisting}

\subsection{Key Expansion Logic}
\begin{enumerate}
	\item Initial Key Setup
	\item Key Expansion based on key size
\end{enumerate}

