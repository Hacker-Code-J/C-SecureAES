\chapter{Block Cipher AES-128}

\section{Overview of Advanced Encryption Standard}

\begin{itemize}
	\item $\text{KeyExpansion}:\binaryfield^{128}\to\binaryfield^{1408}$.
	\item $\text{AddRoundKey}:\binaryfield^{128}\times\binaryfield^{128}\to\binaryfield^{128}$.
	\item $\text{SubBytes}:\binaryfield^{128}\to\binaryfield^{128}$.
	\item $\text{ShiftRows}:\binaryfield^{128}\to\binaryfield^{128}$.
	\item $\text{MixColumns}:\binaryfield^{128}\to\binaryfield^{128}$.
\end{itemize}

\begin{algorithm}[H]
	\caption{Encryption of AES-128}
	
	\KwIn{block $\mathsf{src} \in \binaryfield^{128}$, round-keys $\{rk_i\}_{i=0}^{11}$ ($rk_i \in \binaryfield^{128}$)}
	\KwOut{block $\mathsf{dst} \in \binaryfield^{128}$}
	\BlankLine
	$t \leftarrow \mathsf{src}$\;
	$t \leftarrow \text{AddRoundKey}(t, rk_0)$\;
	\For{$i \leftarrow 1$ \KwTo $9$}{
		$t \leftarrow \text{SubBytes}(t)$\;
		$t \leftarrow \text{ShiftRows}(t)$\;
		$t \leftarrow \text{MixColumns}(t)$\;
		$t \leftarrow \text{AddRoundKey}(t, rk_i)$\;
	}
	$t \leftarrow \text{SubBytes}(t)$\;
	$t \leftarrow \text{ShiftRows}(t)$\;
	$t \leftarrow \text{AddRoundKey}(t, rk_{10})$\;
	$\mathsf{dst} \leftarrow t$\;
	\Return{$\mathsf{dst}$}\;
\end{algorithm}

\newpage
\section{Functions and Constants used in AES}
\subsection{Key Expansion}
\begin{itemize}
	\item \hl{$\text{RotWord}:\binaryfield^{32}\to\binaryfield^{32}$} is defined by \[
	\text{RotWord}\left(X_0\parallel X_1\parallel X_2\parallel X_3\right):=X_1\parallel X_2\parallel X_3\parallel X_0\quad\text{for}\quad X_i\in\binaryfield^8.
	\]
	\begin{lstlisting}[style=C, caption={RotWord rotates the input word left by one byte},captionpos=t]
u32 RotWord(u32 word) {
	return (word << 0x08) | (word >> 0x18);
}
	\end{lstlisting}
	\item \hl{$\text{SubWord}:\binaryfield^{32}\to\binaryfield^{32}$} is defined by \[
	\text{SubWord}(X_0\parallel X_1\parallel X_2\parallel X_3):=s(X_0)\parallel s(X_1)\parallel s(X_2)\parallel s(X_3)\quad\text{for}\quad X_i\in\binaryfield^8.
	\] Here, $s:{\binaryfield^8}\to{\binaryfield^8}$ is the \hyperlink{s-box}{S-box}.
	\begin{lstlisting}[style=C, caption={SubWord applies the S-box to each byte of the input word},captionpos=t]
u32 SubWord(u32 word) {
	return (u32)s_box[word >> 0x18] << 0x18 | 
		(u32)s_box[(word >> 0x10) & 0xFF] << 0x10 | 
		(u32)s_box[(word >> 0x08) & 0xFF] << 0x08 | 
		(u32)s_box[word & 0xFF];
}
	\end{lstlisting}
	\item \hl{Round Constant $\texttt{rCon}$}:
	
	The constant $\texttt{rCon}_i\in\F_{2^8}$ used in generating the $i$-th round key corresponds to the value of $x^{i-1}$ in the binary finite field $\F_{2^8}$ and is as follows:
	\begin{table}[h!]\centering\renewcommand{\arraystretch}{1.5} % Increase row height by 1.5 times
		\begin{tabular*}{\textwidth}{@{\extracolsep{\fill}}c||cccccccccc}
			$i$ & 1 & 2 & 3 & 4 & 5 & 6 & 7 & 8 & 9 & 10\\
			\hline
			$\texttt{Rcon}_i$ & $\texttt{0x01}$ & $\texttt{0x02}$ & $\texttt{0x04}$ & $\texttt{0x08}$ & $\texttt{0x10}$ & $\texttt{0x20}$ & $\texttt{0x40}$ & $\texttt{0x80}$ & $\texttt{0x1b}$ & $\texttt{0x36}$\\
		\end{tabular*}
	\end{table}
	\begin{lstlisting}[style=C, caption={rCon Array Declaration},captionpos=t]
static const u32 rCon[10] = {
	0x01000000, 0x02000000, 0x04000000, 0x08000000,
	0x10000000, 0x20000000, 0x40000000, 0x80000000,
	0x1b000000, 0x36000000
};
	\end{lstlisting}
\end{itemize}
\begin{algorithm}[H]
	\caption{Key Schedule (AES-128)}
	
	\KwIn{key \( k = (k_0, \dots, k_{15}) \) \( (k_i \in \binaryfield^8) \)\tcp*{$k\in\binaryfield^{128}$ is 16-byte}}
	\KwOut{round-keys \( \{rk_i\}_{i=0}^{43} \) \( (rk_i \in \binaryfield^{32}) \)\tcp*{$\set{rk_i}_{i=0}^{43}\in\binaryfield^{1408}$ is 176-byte}}
	\BlankLine
	\( rk_0 \leftarrow k_0\parallel k_1\parallel k_2 \parallel k_3 \)\;
	\( rk_1 \leftarrow k_4\parallel k_5\parallel k_6 \parallel k_7 \)\;
	\( rk_2 \leftarrow k_8\parallel k_9\parallel k_{10} \parallel k_{11} \)\;
	\( rk_3 \leftarrow k_{12}\parallel k_{13}\parallel k_{14} \parallel k_{15} \)\;
	\For{\( i = 4 \) \KwTo \( 43 \)}{
		\( t \leftarrow rk_{i-1} \)\;
		\If{\( i \bmod 4 = 0 \)}{
			\Comment{$\text{SubWord}\circ\text{RotWord}:\binaryfield^{32}\to\binaryfield^{32}$}
			\( t \leftarrow \text{RotWord}(t) \)\;
			\( t \leftarrow \text{SubWord}(t) \)\;
			\( t \leftarrow t \oplus (\texttt{rCon}_{i/4}\parallel\texttt{0x00}\parallel\texttt{0x00}\parallel\texttt{0x00}) \)\;
		}
		\( rk_i \leftarrow rk_{i-4} \oplus t \)\;
	}
\end{algorithm}
\vspace{24pt}
\begin{lstlisting}[style=C, caption={AES Key Expansion},captionpos=t]
void KeyExpansion(const u8* key, u32* rKey) {
	u32 temp;
	int i = 0;
	
	// Copy the input key to the first round key
	while (i < 4) {
		rKey[i] = (u32)key[4*i] << 0x18 | 
		(u32)key[4*i+1] << 0x10 | 
		(u32)key[4*i+2] << 0x08 | 
		(u32)key[4*i+3];
		i++;
	}
	
	i = 4;
	
	// Generate the remaining round keys
	while (i < 44) {
		temp = rKey[i-1];
		if (i % 4 == 0) {
			temp = SubWord(RotWord(temp)) ^ rCon[i/4-1];
		}
		rKey[i] = rKey[i-4] ^ temp;
		i++;
	}
}
\end{lstlisting}

\newpage
\subsection{AddRoundKey}
\begin{itemize}
	\item $\text{AddRoundKey}:\binaryfield^{128}\times\binaryfield^{128}\to\binaryfield^{128}$ is defined by \[
	\text{AddRoundKey}\left(\set{X_i}_{i=0}^{15},\set{rk_i}_{i=0}^{15}\right)
	\] 
\end{itemize}
\subsection{SubBytes}
\subsection{ShiftRows}
\subsection{MiColums}

\newpage
\section{Code Structure}
\begin{enumerate}
	\item Rcon Array Declaration
	\item Function Definition
	\item Variable Declarations and Initial Checks
	\item Key Expansion Logic
\end{enumerate}

\section{Detailed Analysis}
\subsection{Rcon Array Declaration}


\subsection{Function Definition}
\begin{lstlisting}[language=C]
	int AES_set_encrypt_key(const unsigned char *userKey, const int bits, AES_KEY *key);
\end{lstlisting}

\subsection{Variable Declarations and Initial Checks}
\begin{lstlisting}[language=C]
	u32 *rk;
	int i = 0;
	u32 temp;
	if (!userKey || !key)
	return -1;
	if (bits != 128 && bits != 192 && bits != 256)
	return -2;
\end{lstlisting}

\subsection{Key Expansion Logic}
\begin{enumerate}
	\item Initial Key Setup
	\item Key Expansion based on key size
\end{enumerate}

