\section*{Acknowledgements}

\begin{note}[\textbf{XOR Operation and Modular Reduction in \( GF(2^n) \)}]
	In the context of Galois Field \( GF(2^n) \), particularly in binary polynomial arithmetic, the XOR operation is equivalent to addition and also plays a crucial role in modular reduction. We explore this equivalence through the principles of field theory and polynomial arithmetic.
	\begin{itemize}
		\item \textbf{Field Properties:}
		
		A Galois Field, \( GF(p^n) \), is a finite field that contains a finite number of elements, where
		\begin{itemize}
			\item \( p \) is a prime number (base of the field) and
			\item \( n \) is a positive integer (degree of the field).
		\end{itemize}
		For the binary field \( GF(2^n) \), \( p = 2 \), which implies that every element in this field is either 0 or 1.
		\vspace{8pt}
		\item \textbf{Addition in \( GF(2^n) \):}
		
		In \( GF(2^n) \), the addition of two elements is performed modulo 2. For any two elements \( a, b \in GF(2^n) \), the addition is defined as:
		\[ a + b = a \oplus b\]
		Since \( 2 \) is the base of the field, the addition wraps around upon reaching 2, which is effectively what the XOR operation does.
		\vspace{8pt}
		\item \textbf{Polynomial Representation:}
		
		Elements in \( GF(2^n) \) can be represented as polynomials where each coefficient is in \( GF(2)=\set{0,1} \). A general element can be written as:
		\[ a(x) = a_{n-1}x^{n-1} + a_{n-2}x^{n-2} + \cdots + a_1x + a_0 \]
		where \( a_i \in \{0, 1\} \) for all \( i \).
		\vspace{8pt}
		\item \textbf{Modular Reduction:}
		
		Modular reduction in \( GF(2^n) \) involves reducing a polynomial by a fixed irreducible polynomial of degree \( n \), ensuring that the result remains within the field. Let \( m(x) \) be the irreducible polynomial. The reduction of a polynomial \( f(x) \) is given by:
		$ f(x) \mod m(x)$
		\vspace{8pt}
		\item \textbf{XOR as Modular Reduction:}
		
		During modular reduction, the subtraction used in polynomial division becomes XOR, because subtraction and addition are the same in \( GF(2) \). Therefore, reducing a polynomial \( f(x) \) by \( m(x) \) is effectively performed using XOR on the coefficients of corresponding terms.
		
		For example, if \( f(x) \) has a term \( x^k \) where \( k \geq n \), and \( m(x) \) has a term \( x^k \), then reducing \( f(x) \) by \( m(x) \) involves XORing the coefficients of \( x^k \) in \( f(x) \) and \( m(x) \), effectively eliminating the \( x^k \) term in \( f(x) \).
	\end{itemize}
	
	In summary, the XOR operation becomes equivalent to both addition and modular reduction in \( GF(2^n) \) due to the binary nature of the field. This equivalence simplifies polynomial arithmetic in binary fields, making it a cornerstone of operations in cryptographic algorithms.
\end{note}

%\section*{Functional Programming}
%
%Functional programming is a paradigm based on mathematical functions, emphasizing expressions and declarations over statements. Key characteristics include:
%
%\begin{itemize}
%	\item \textbf{Immutability}: Data is immutable, which leads to fewer side effects and more predictable code.
%	\item \textbf{First-Class and Higher-Order Functions}: Functions are treated as first-class citizens, allowing them to be passed around, returned, and assigned to variables.
%	\item \textbf{Statelessness}: It avoids shared state and relies on immutable data and pure functions.
%	\item \textbf{Examples}: Haskell, Clojure, and parts of JavaScript, Python, and Scala.
%\end{itemize}
%
%\section*{Imperative Programming}
%
%Imperative programming focuses on a sequence of commands for the computer to perform. Its main features include:
%
%\begin{itemize}
%	\item \textbf{State and Mutability}: Variables in imperative programming can be changed, leading to mutable program states.
%	\item \textbf{Commands and Control Structures}: It utilizes loops, conditionals, and instructions for flow control.
%	\item \textbf{Direct Manipulation of Memory or State}: Often involves changing the program’s state or memory directly.
%	\item \textbf{Examples}: C, C++, Java, and Python.
%\end{itemize}
%
%\section*{Key Differences}
%
%The fundamental distinctions between functional and imperative programming lie in:
%
%\begin{itemize}
%	\item \textbf{Approach to State and Data}: Functional programming uses immutable data, contrasting with the mutable data in imperative programming.
%	\item \textbf{Flow Control}: Function calls and recursion are primary in functional programming, whereas loops and control structures are used in imperative programming.
%	\item \textbf{Side Effects}: Functional programming minimizes side effects, a contrast to the often side-effect-laden imperative programming.
%	\item \textbf{Conciseness and Expressiveness}: Functional programming can be more concise for tasks involving data transformations or concurrent processing.
%	\item \textbf{Learning Curve}: Functional programming often requires a different mindset, focusing on what to solve rather than how, and can have a steeper learning curve.
%\end{itemize}
%
%In contemporary software development, many languages blend elements from both paradigms, offering flexibility and choice to the programmer.